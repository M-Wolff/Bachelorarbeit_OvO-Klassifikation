\chapter{Methodik}
\label{ch:methodik}
\todo{deutlich machen, was ich anders gemacht habe als Pawara und wieso}
\todo{generelles Verfahren in eigenen Worten beschreiben}

\section{OvO-Kodierung}
\label{ch:methodik_kodierung}
\todo{Umsetzung: OvO Matrix, Kodierung, Klassifizierung, eigene Loss-Funktion, softmax vs tanh}

\subsection{Alternative Kodierungsmethoden}
\todo{alternative Kodierungsmöglichkeiten beschreiben (s. Paper, z.B. Error correcting codes)}


\section{Datensätze}
\label{ch:methodik_datensaetze}
\todo{Datensätze (cifar10)}
\todo{kaputte / falsche Datensätze auf Pawara Homepage (unsaubere 5-fold, teilweise 70/30)}
\todo{TF Versionen: Warum beide benutzt?}
\todo{eigene 5- / 3-fold cross validation}

\section{Trainingsparameter}
\label{ch:methodik_parameter}

\subsection{Netztypen}
\todo{Netze, Pawara's Veränderung an Inception}
\todo{Implementation der Netze in den verschiedenen Frameworks vergleichen}

\subsection{Trainsize}
\todo{deterministische Datensatz-Splits}

\subsection{Klassenanzahl}

\subsection{vortrainierte Gewichte}
\todo{Epochen, Learningrate}


\section{Ausführung der Jobs auf Palma II}
\label{ch:methodik_palma}
\todo{Hardware auf Palma, Einteilung in Jobs, Logging, benutzte Grafikkarten, CPU Limits}

