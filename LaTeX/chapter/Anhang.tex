% Platz vor Chapter-Überschrift verringern (damit auch auf die erste Seite 2 Tabellen untereinander passen)
\makeatletter
\let\savedchap\@makechapterhead
\def\@makechapterhead{\vspace*{-2cm}\savedchap}
\makeatletter
% Seitenränder verringern (unten und oben)
\newgeometry{a4paper,left=35mm,right=35mm,top=5mm,bottom=5mm, includeheadfoot}


\chapter{Tabellen mit Ergebnissen}
\label{ch:Anhang_Tabellen}
\section{Tensorflow 1.13.1}

\subsection{ResNet-50 Scratch}
\begin{figure}[H]
\resizebox{\textwidth}{!}{
\begin{tabular}{|c|c|c|}
\multicolumn{3}{c}{agrilplant3} \\
\hline
\hline
\scriptsize{Train Prozent} & OvO & OvA \\
\hline 10 & 78.67 \plm 1.45 & 77.67 \plm 2.84\\
20 & 84.44 \plm 1.42 & 85.00 \plm 4.04\\
50 & 96.00 \plm 1.86 & 94.22 \plm 2.17\\
80 & 97.00 \plm 1.01 & 96.67 \plm 0.88\\
100 & 98.33 \plm 0.68 & 97.89 \plm 0.61\\
\hline 
\end{tabular}

\begin{tabular}{|c|c|}
\multicolumn{2}{c}{agrilplant5} \\
\hline
\hline
OvO & OvA \\
\hline 75.93 \plm 2.20 & 74.13 \plm 4.17\\
87.40 \plm 3.72 & 85.40 \plm 4.41\\
96.80 \plm 1.07 & 95.60 \plm 1.89\\
97.67 \plm 0.97 & 97.60 \plm 0.89\\
98.27 \plm 0.55 & 97.73 \plm 0.95\\
\hline 
\end{tabular}

\begin{tabular}{|c|c|}
\multicolumn{2}{c}{agrilplant10} \\
\hline
\hline
OvO & OvA \\
\hline 74.37 \plm 3.58 & 69.93 \plm 5.32\\
84.80 \plm 2.25 & 83.53 \plm 2.54\\
92.47 \plm 1.18 & 92.47 \plm 0.36\\
95.33 \plm 0.86 & 95.33 \plm 0.67\\
96.03 \plm 0.62 & 95.77 \plm 0.61\\
\hline 
\end{tabular}
}%
\vspace{2mm}
\resizebox{\textwidth}{!}{
\begin{tabular}{|c|c|c|}
\multicolumn{3}{c}{tropic3} \\
\hline
\hline
\scriptsize{Train Prozent} & OvO & OvA \\
\hline 10 & 88.96 \plm 2.78 & 87.75 \plm 4.49\\
20 & 92.72 \plm 1.48 & 91.51 \plm 1.28\\
50 & 98.18 \plm 1.83 & 97.69 \plm 1.32\\
80 & 98.06 \plm 0.79 & 97.93 \plm 1.40\\
100 & 99.03 \plm 0.92 & 98.91 \plm 0.90\\
\hline 
\end{tabular}

\begin{tabular}{|c|c|}
\multicolumn{2}{c}{tropic5} \\
\hline
\hline
OvO & OvA \\
\hline 78.72 \plm 2.04 & 76.90 \plm 3.63\\
89.03 \plm 3.15 & 86.86 \plm 3.51\\
96.08 \plm 1.96 & 96.59 \plm 1.04\\
97.90 \plm 1.41 & 98.77 \plm 1.16\\
99.13 \plm 0.41 & 98.62 \plm 0.65\\
\hline 
\end{tabular}

\begin{tabular}{|c|c|}
\multicolumn{2}{c}{tropic10} \\
\hline
\hline
OvO & OvA \\
\hline 76.48 \plm 1.08 & 67.67 \plm 4.78\\
87.51 \plm 1.41 & 82.66 \plm 1.80\\
95.15 \plm 0.38 & 94.25 \plm 1.32\\
97.14 \plm 0.56 & 97.73 \plm 0.33\\
98.08 \plm 0.59 & 97.77 \plm 0.75\\
\hline 
\end{tabular}

\begin{tabular}{|c|c|}
\multicolumn{2}{c}{tropic20} \\
\hline
\hline
OvO & OvA \\
\hline 73.50 \plm 1.51 & 70.93 \plm 1.82\\
84.55 \plm 2.16 & 81.99 \plm 2.54\\
93.16 \plm 2.02 & 94.05 \plm 1.61\\
96.21 \plm 0.80 & 96.30 \plm 0.75\\
96.74 \plm 0.93 & 97.78 \plm 0.59\\
\hline 
\end{tabular}
}%
\vspace{2mm}
\resizebox{\textwidth}{!}{
\begin{tabular}{|c|c|c|}
\multicolumn{3}{c}{swedishLeaves3folds3} \\
\hline
\hline
\scriptsize{Train Prozent} & OvO & OvA \\
\hline 10 & 58.67 \plm 7.33 & 55.78 \plm 7.78\\
20 & 67.11 \plm 6.74 & 69.11 \plm 2.69\\
50 & 85.56 \plm 1.02 & 87.33 \plm 1.76\\
80 & 91.56 \plm 1.02 & 90.89 \plm 1.92\\
100 & 94.00 \plm 1.33 & 94.22 \plm 2.52\\
\hline 
\end{tabular}

\begin{tabular}{|c|c|}
\multicolumn{2}{c}{swedishLeaves3folds5} \\
\hline
\hline
OvO & OvA \\
\hline 58.53 \plm 11.21 & 44.13 \plm 9.75\\
84.40 \plm 2.62 & 82.27 \plm 5.83\\
93.33 \plm 1.29 & 91.73 \plm 1.67\\
95.73 \plm 2.81 & 94.80 \plm 1.06\\
97.07 \plm 0.46 & 95.47 \plm 0.83\\
\hline 
\end{tabular}

\begin{tabular}{|c|c|}
\multicolumn{2}{c}{swedishLeaves3folds10} \\
\hline
\hline
OvO & OvA \\
\hline 76.93 \plm 4.32 & 78.13 \plm 4.21\\
87.40 \plm 2.08 & 89.00 \plm 1.11\\
93.13 \plm 1.45 & 92.07 \plm 2.50\\
95.07 \plm 1.81 & 94.00 \plm 1.97\\
96.80 \plm 0.69 & 96.53 \plm 0.90\\
\hline 
\end{tabular}

\begin{tabular}{|c|c|}
\multicolumn{2}{c}{swedishLeaves3folds15} \\
\hline
\hline
OvO & OvA \\
\hline 65.82 \plm 11.74 & 71.20 \plm 4.80\\
91.60 \plm 1.16 & 88.40 \plm 3.20\\
93.16 \plm 1.44 & 93.38 \plm 1.16\\
96.84 \plm 0.20 & 94.80 \plm 3.84\\
97.11 \plm 0.15 & 96.84 \plm 0.73\\
\hline 
\end{tabular}
}%
\vspace{2mm}
\resizebox{\textwidth}{!}{
\begin{tabular}{|c|c|c|}
\multicolumn{3}{c}{pawara-tropic10} \\
\hline
\hline
\scriptsize{Train Prozent} & OvO & OvA \\
\hline 10 & 69.74 \plm 3.69 & 64.42 \plm 3.61\\
20 & 83.54 \plm 1.48 & 78.06 \plm 0.96\\
50 & 93.66 \plm 1.81 & 92.90 \plm 1.17\\
80 & 96.66 \plm 0.68 & 97.08 \plm 0.76\\
100 & 97.94 \plm 0.55 & 97.81 \plm 0.60\\
\hline 
\end{tabular}

\begin{tabular}{|c|c|}
\multicolumn{2}{c}{pawara-monkey10} \\
\hline
\hline
OvO & OvA \\
\hline 47.81 \plm 2.62 & 42.57 \plm 2.77\\
59.33 \plm 2.78 & 57.29 \plm 2.62\\
73.38 \plm 8.29 & 71.46 \plm 3.46\\
86.79 \plm 1.53 & 86.42 \plm 1.58\\
89.05 \plm 1.18 & 89.12 \plm 1.85\\
\hline 
\end{tabular}

\begin{tabular}{|c|c|}
\multicolumn{2}{c}{pawara-umonkey10} \\
\hline
\hline
OvO & OvA \\
\hline 31.97 \plm 1.63 & 32.56 \plm 3.61\\
43.64 \plm 3.75 & 38.83 \plm 1.76\\
60.01 \plm 3.48 & 56.72 \plm 4.37\\
67.67 \plm 0.96 & 67.01 \plm 1.96\\
69.65 \plm 5.30 & 68.47 \plm 3.21\\
\hline 
\end{tabular}

\begin{tabular}{|c|c|}
\multicolumn{2}{c}{cifar10} \\
\hline
\hline
OvO & OvA \\
\hline 76.50 \plm 1.08 & 79.09 \plm 1.44\\
83.91 \plm 0.86 & 85.41 \plm 0.67\\
90.16 \plm 0.50 & 90.68 \plm 0.47\\
92.17 \plm 0.54 & 92.70 \plm 0.66\\
92.84 \plm 0.93 & 93.87 \plm 0.18\\
\hline 
\end{tabular}
}%

\end{figure}

\subsection{ResNet-50 Finetune}
\begin{figure}[H]
\resizebox{\textwidth}{!}{
\begin{tabular}{|c|c|c|}
\multicolumn{3}{c}{agrilplant3} \\
\hline
\hline
\scriptsize{Train Prozent} & OvO & OvA \\
\hline 10 & 99.00 \plm 0.61 & 99.22 \plm 0.63\\
20 & 99.67 \plm 0.50 & 99.44 \plm 0.56\\
50 & 99.89 \plm 0.25 & 99.56 \plm 0.25\\
80 & 100.00 \plm 0.00 & 99.89 \plm 0.25\\
100 & 100.00 \plm 0.00 & 100.00 \plm 0.00\\
\hline 
\end{tabular}

\begin{tabular}{|c|c|}
\multicolumn{2}{c}{agrilplant5} \\
\hline
\hline
OvO & OvA \\
\hline 95.40 \plm 1.36 & 94.73 \plm 1.55\\
98.33 \plm 0.91 & 96.80 \plm 0.77\\
99.13 \plm 0.38 & 99.27 \plm 0.76\\
99.60 \plm 0.28 & 99.73 \plm 0.28\\
99.40 \plm 0.55 & 99.87 \plm 0.18\\
\hline 
\end{tabular}

\begin{tabular}{|c|c|}
\multicolumn{2}{c}{agrilplant10} \\
\hline
\hline
OvO & OvA \\
\hline 92.03 \plm 1.28 & 92.67 \plm 0.73\\
95.23 \plm 0.78 & 95.77 \plm 0.84\\
96.90 \plm 0.45 & 97.70 \plm 0.77\\
97.87 \plm 0.41 & 98.67 \plm 0.51\\
98.07 \plm 0.42 & 98.67 \plm 0.42\\
\hline 
\end{tabular}
}%
\vspace{2mm}
\resizebox{\textwidth}{!}{
\begin{tabular}{|c|c|c|}
\multicolumn{3}{c}{tropic3} \\
\hline
\hline
\scriptsize{Train Prozent} & OvO & OvA \\
\hline 10 & 98.30 \plm 0.79 & 96.60 \plm 1.85\\
20 & 99.51 \plm 0.27 & 99.27 \plm 0.27\\
50 & 99.76 \plm 0.33 & 99.39 \plm 0.61\\
80 & 99.27 \plm 0.27 & 100.00 \plm 0.00\\
100 & 99.88 \plm 0.27 & 99.76 \plm 0.54\\
\hline 
\end{tabular}

\begin{tabular}{|c|c|}
\multicolumn{2}{c}{tropic5} \\
\hline
\hline
OvO & OvA \\
\hline 96.37 \plm 1.03 & 95.72 \plm 1.10\\
98.76 \plm 1.14 & 98.62 \plm 0.83\\
99.56 \plm 0.30 & 99.35 \plm 0.16\\
99.56 \plm 0.16 & 99.85 \plm 0.20\\
99.71 \plm 0.30 & 99.78 \plm 0.20\\
\hline 
\end{tabular}

\begin{tabular}{|c|c|}
\multicolumn{2}{c}{tropic10} \\
\hline
\hline
OvO & OvA \\
\hline 92.92 \plm 1.08 & 94.87 \plm 0.82\\
96.48 \plm 0.91 & 97.69 \plm 0.45\\
98.86 \plm 0.42 & 99.45 \plm 0.40\\
98.98 \plm 0.32 & 99.61 \plm 0.48\\
99.41 \plm 0.37 & 99.73 \plm 0.30\\
\hline 
\end{tabular}

\begin{tabular}{|c|c|}
\multicolumn{2}{c}{tropic20} \\
\hline
\hline
OvO & OvA \\
\hline 91.36 \plm 0.81 & 94.24 \plm 1.13\\
95.81 \plm 0.63 & 97.48 \plm 0.50\\
97.99 \plm 0.53 & 98.82 \plm 0.49\\
98.90 \plm 0.13 & 99.34 \plm 0.24\\
99.05 \plm 0.35 & 99.66 \plm 0.11\\
\hline 
\end{tabular}
}%
\vspace{2mm}
\resizebox{\textwidth}{!}{
\begin{tabular}{|c|c|c|}
\multicolumn{3}{c}{swedishLeaves3folds3} \\
\hline
\hline
\scriptsize{Train Prozent} & OvO & OvA \\
\hline 10 & 84.22 \plm 6.58 & 86.00 \plm 1.76\\
20 & 92.44 \plm 3.91 & 91.11 \plm 2.78\\
50 & 95.33 \plm 3.53 & 94.67 \plm 4.81\\
80 & 99.78 \plm 0.38 & 97.33 \plm 1.33\\
100 & 98.44 \plm 0.38 & 98.67 \plm 0.67\\
\hline 
\end{tabular}

\begin{tabular}{|c|c|}
\multicolumn{2}{c}{swedishLeaves3folds5} \\
\hline
\hline
OvO & OvA \\
\hline 93.47 \plm 0.61 & 92.53 \plm 1.51\\
96.13 \plm 3.45 & 96.93 \plm 2.95\\
99.07 \plm 0.23 & 97.60 \plm 2.12\\
99.20 \plm 0.80 & 99.73 \plm 0.23\\
99.60 \plm 0.40 & 99.73 \plm 0.23\\
\hline 
\end{tabular}

\begin{tabular}{|c|c|}
\multicolumn{2}{c}{swedishLeaves3folds10} \\
\hline
\hline
OvO & OvA \\
\hline 94.20 \plm 2.46 & 93.13 \plm 1.70\\
97.73 \plm 0.23 & 96.40 \plm 0.20\\
97.13 \plm 1.70 & 97.13 \plm 2.39\\
99.20 \plm 0.72 & 99.47 \plm 0.50\\
99.73 \plm 0.31 & 99.80 \plm 0.20\\
\hline 
\end{tabular}

\begin{tabular}{|c|c|}
\multicolumn{2}{c}{swedishLeaves3folds15} \\
\hline
\hline
OvO & OvA \\
\hline 96.98 \plm 0.73 & 95.64 \plm 1.26\\
98.40 \plm 0.46 & 97.87 \plm 0.81\\
99.16 \plm 0.60 & 98.89 \plm 0.28\\
99.56 \plm 0.28 & 99.69 \plm 0.20\\
99.56 \plm 0.31 & 99.82 \plm 0.20\\
\hline 
\end{tabular}
}%
\vspace{2mm}
\resizebox{\textwidth}{!}{
\begin{tabular}{|c|c|c|}
\multicolumn{3}{c}{pawara-tropic10} \\
\hline
\hline
\scriptsize{Train Prozent} & OvO & OvA \\
\hline 10 & 93.69 \plm 1.49 & 94.60 \plm 1.04\\
20 & 95.43 \plm 1.51 & 97.08 \plm 1.06\\
50 & 98.38 \plm 0.51 & 99.43 \plm 0.20\\
80 & 99.09 \plm 0.28 & 99.63 \plm 0.11\\
100 & 99.32 \plm 0.34 & 99.74 \plm 0.21\\
\hline 
\end{tabular}

\begin{tabular}{|c|c|}
\multicolumn{2}{c}{pawara-monkey10} \\
\hline
\hline
OvO & OvA \\
\hline 89.41 \plm 1.93 & 90.95 \plm 1.21\\
93.43 \plm 2.07 & 93.36 \plm 1.55\\
95.33 \plm 1.43 & 95.47 \plm 1.12\\
96.93 \plm 0.62 & 97.67 \plm 1.22\\
97.30 \plm 0.98 & 97.66 \plm 0.41\\
\hline 
\end{tabular}

\begin{tabular}{|c|c|}
\multicolumn{2}{c}{pawara-umonkey10} \\
\hline
\hline
OvO & OvA \\
\hline 73.80 \plm 3.01 & 70.31 \plm 3.36\\
84.38 \plm 1.97 & 81.24 \plm 2.34\\
90.58 \plm 1.56 & 88.32 \plm 2.03\\
91.09 \plm 2.09 & 91.15 \plm 2.40\\
93.28 \plm 0.45 & 91.75 \plm 2.04\\
\hline 
\end{tabular}
}%

\end{figure}

\newpage

\subsection{InceptionV3 Scratch}
\begin{figure}[H]
\resizebox{\textwidth}{!}{
\begin{tabular}{|c|c|c|}
\multicolumn{3}{c}{agrilplant3} \\
\hline
\hline
\scriptsize{Train Prozent} & OvO & OvA \\
\hline 10 & 80.44 \plm 2.06 & 82.78 \plm 4.46\\
20 & 89.11 \plm 1.34 & 87.89 \plm 1.90\\
50 & 96.89 \plm 1.01 & 97.67 \plm 0.72\\
80 & 98.33 \plm 0.88 & 98.33 \plm 0.79\\
100 & 98.11 \plm 1.15 & 98.44 \plm 0.25\\
\hline 
\end{tabular}

\begin{tabular}{|c|c|}
\multicolumn{2}{c}{agrilplant5} \\
\hline
\hline
OvO & OvA \\
\hline 82.47 \plm 1.79 & 80.73 \plm 2.62\\
90.20 \plm 1.28 & 88.73 \plm 3.95\\
97.33 \plm 0.53 & 97.53 \plm 0.96\\
98.47 \plm 0.45 & 98.20 \plm 0.65\\
98.87 \plm 0.77 & 98.93 \plm 0.55\\
\hline 
\end{tabular}

\begin{tabular}{|c|c|}
\multicolumn{2}{c}{agrilplant10} \\
\hline
\hline
OvO & OvA \\
\hline 78.60 \plm 2.68 & 74.47 \plm 2.88\\
87.43 \plm 1.64 & 86.90 \plm 1.72\\
94.07 \plm 0.56 & 94.07 \plm 0.56\\
96.40 \plm 0.71 & 96.07 \plm 0.49\\
96.47 \plm 0.72 & 97.23 \plm 0.52\\
\hline 
\end{tabular}
}%
\vspace{2mm}
\resizebox{\textwidth}{!}{
\begin{tabular}{|c|c|c|}
\multicolumn{3}{c}{tropic3} \\
\hline
\hline
\scriptsize{Train Prozent} & OvO & OvA \\
\hline 10 & 88.47 \plm 2.90 & 88.11 \plm 4.46\\
20 & 93.20 \plm 2.52 & 92.35 \plm 1.52\\
50 & 97.57 \plm 0.96 & 97.45 \plm 0.80\\
80 & 98.18 \plm 1.13 & 98.30 \plm 0.79\\
100 & 98.79 \plm 0.43 & 98.42 \plm 0.93\\
\hline 
\end{tabular}

\begin{tabular}{|c|c|}
\multicolumn{2}{c}{tropic5} \\
\hline
\hline
OvO & OvA \\
\hline 79.16 \plm 2.78 & 77.85 \plm 3.15\\
91.14 \plm 2.87 & 87.95 \plm 2.41\\
97.24 \plm 1.24 & 96.95 \plm 0.91\\
99.06 \plm 0.55 & 98.40 \plm 0.48\\
99.27 \plm 0.57 & 98.48 \plm 1.04\\
\hline 
\end{tabular}

\begin{tabular}{|c|c|}
\multicolumn{2}{c}{tropic10} \\
\hline
\hline
OvO & OvA \\
\hline 78.75 \plm 1.34 & 72.33 \plm 1.55\\
88.06 \plm 1.91 & 84.11 \plm 1.66\\
97.22 \plm 0.81 & 95.54 \plm 1.48\\
98.75 \plm 0.33 & 98.12 \plm 0.78\\
98.98 \plm 0.38 & 98.51 \plm 0.26\\
\hline 
\end{tabular}

\begin{tabular}{|c|c|}
\multicolumn{2}{c}{tropic20} \\
\hline
\hline
OvO & OvA \\
\hline 74.43 \plm 1.37 & 69.64 \plm 1.68\\
86.05 \plm 1.00 & 82.34 \plm 1.31\\
96.11 \plm 0.79 & 94.41 \plm 1.02\\
97.52 \plm 0.46 & 97.61 \plm 0.68\\
98.71 \plm 0.36 & 98.58 \plm 0.07\\
\hline 
\end{tabular}
}%
\vspace{2mm}
\resizebox{\textwidth}{!}{
\begin{tabular}{|c|c|c|}
\multicolumn{3}{c}{swedishLeaves3folds3} \\
\hline
\hline
\scriptsize{Train Prozent} & OvO & OvA \\
\hline 10 & 69.33 \plm 7.57 & 61.78 \plm 9.71\\
20 & 67.33 \plm 15.01 & 71.78 \plm 3.15\\
50 & 87.78 \plm 2.78 & 85.11 \plm 1.02\\
80 & 94.89 \plm 1.02 & 93.33 \plm 1.15\\
100 & 94.67 \plm 1.76 & 93.33 \plm 2.40\\
\hline 
\end{tabular}

\begin{tabular}{|c|c|}
\multicolumn{2}{c}{swedishLeaves3folds5} \\
\hline
\hline
OvO & OvA \\
\hline 66.27 \plm 2.84 & 64.27 \plm 8.26\\
88.40 \plm 1.39 & 85.60 \plm 3.17\\
93.20 \plm 3.12 & 92.00 \plm 1.06\\
95.87 \plm 1.01 & 94.13 \plm 0.61\\
98.80 \plm 0.69 & 96.13 \plm 0.92\\
\hline 
\end{tabular}

\begin{tabular}{|c|c|}
\multicolumn{2}{c}{swedishLeaves3folds10} \\
\hline
\hline
OvO & OvA \\
\hline 82.47 \plm 5.75 & 79.67 \plm 4.96\\
90.27 \plm 2.80 & 90.53 \plm 1.14\\
94.07 \plm 0.83 & 91.33 \plm 0.83\\
97.00 \plm 0.40 & 96.67 \plm 0.12\\
97.53 \plm 1.10 & 97.33 \plm 0.42\\
\hline 
\end{tabular}

\begin{tabular}{|c|c|}
\multicolumn{2}{c}{swedishLeaves3folds15} \\
\hline
\hline
OvO & OvA \\
\hline 83.64 \plm 3.81 & 79.96 \plm 2.65\\
92.09 \plm 3.35 & 88.80 \plm 1.67\\
94.27 \plm 0.61 & 92.93 \plm 1.62\\
98.09 \plm 0.38 & 98.09 \plm 0.96\\
98.13 \plm 0.48 & 98.13 \plm 0.13\\
\hline 
\end{tabular}
}%
\vspace{2mm}
\resizebox{\textwidth}{!}{
\begin{tabular}{|c|c|c|}
\multicolumn{3}{c}{pawara-tropic10} \\
\hline
\hline
\scriptsize{Train Prozent} & OvO & OvA \\
\hline 10 & 73.52 \plm 2.98 & 67.28 \plm 2.98\\
20 & 87.06 \plm 2.42 & 81.40 \plm 1.60\\
50 & 96.61 \plm 0.64 & 94.42 \plm 1.47\\
80 & 98.51 \plm 0.60 & 97.37 \plm 0.77\\
100 & 98.72 \plm 0.14 & 98.04 \plm 0.51\\
\hline 
\end{tabular}

\begin{tabular}{|c|c|}
\multicolumn{2}{c}{pawara-monkey10} \\
\hline
\hline
OvO & OvA \\
\hline 52.34 \plm 1.86 & 45.70 \plm 3.21\\
66.56 \plm 4.18 & 61.10 \plm 4.23\\
78.75 \plm 5.41 & 78.54 \plm 1.99\\
90.08 \plm 2.52 & 89.71 \plm 1.81\\
91.90 \plm 1.94 & 91.89 \plm 2.00\\
\hline 
\end{tabular}

\begin{tabular}{|c|c|}
\multicolumn{2}{c}{pawara-umonkey10} \\
\hline
\hline
OvO & OvA \\
\hline 37.51 \plm 3.27 & 34.73 \plm 3.44\\
46.87 \plm 4.88 & 45.47 \plm 5.71\\
61.47 \plm 6.48 & 60.50 \plm 6.18\\
71.97 \plm 2.76 & 68.25 \plm 1.79\\
75.83 \plm 1.95 & 72.19 \plm 1.53\\
\hline 
\end{tabular}
}%

\end{figure}
\subsection{InceptionV3 Finetune}
\begin{figure}[H]
\resizebox{\textwidth}{!}{
\begin{tabular}{|c|c|c|}
\multicolumn{3}{c}{agrilplant3} \\
\hline
\hline
\scriptsize{Train Prozent} & OvO & OvA \\
\hline 10 & 99.44 \plm 0.39 & 99.33 \plm 0.46\\
20 & 99.67 \plm 0.50 & 99.67 \plm 0.30\\
50 & 99.89 \plm 0.25 & 99.78 \plm 0.30\\
80 & 99.89 \plm 0.25 & 99.78 \plm 0.30\\
100 & 100.00 \plm 0.00 & 100.00 \plm 0.00\\
\hline 
\end{tabular}

\begin{tabular}{|c|c|}
\multicolumn{2}{c}{agrilplant5} \\
\hline
\hline
OvO & OvA \\
\hline 96.27 \plm 2.33 & 95.60 \plm 0.98\\
98.00 \plm 0.58 & 98.13 \plm 0.90\\
98.93 \plm 0.55 & 99.40 \plm 0.55\\
99.27 \plm 0.37 & 99.53 \plm 0.56\\
99.53 \plm 0.56 & 99.73 \plm 0.28\\
\hline 
\end{tabular}

\begin{tabular}{|c|c|}
\multicolumn{2}{c}{agrilplant10} \\
\hline
\hline
OvO & OvA \\
\hline 92.73 \plm 0.43 & 94.30 \plm 0.84\\
95.17 \plm 0.47 & 96.70 \plm 0.59\\
97.67 \plm 0.85 & 98.50 \plm 0.95\\
98.13 \plm 0.55 & 98.53 \plm 0.59\\
98.17 \plm 0.33 & 98.53 \plm 0.62\\
\hline 
\end{tabular}
}%
\vspace{2mm}
\resizebox{\textwidth}{!}{
\begin{tabular}{|c|c|c|}
\multicolumn{3}{c}{tropic3} \\
\hline
\hline
\scriptsize{Train Prozent} & OvO & OvA \\
\hline 10 & 99.03 \plm 0.92 & 97.57 \plm 1.21\\
20 & 99.88 \plm 0.27 & 99.39 \plm 0.43\\
50 & 99.64 \plm 0.33 & 99.76 \plm 0.33\\
80 & 99.88 \plm 0.27 & 100.00 \plm 0.00\\
100 & 99.64 \plm 0.33 & 100.00 \plm 0.00\\
\hline 
\end{tabular}

\begin{tabular}{|c|c|}
\multicolumn{2}{c}{tropic5} \\
\hline
\hline
OvO & OvA \\
\hline 97.31 \plm 1.64 & 98.04 \plm 1.22\\
99.42 \plm 0.32 & 99.06 \plm 0.49\\
99.64 \plm 0.36 & 99.49 \plm 0.32\\
99.64 \plm 0.44 & 99.93 \plm 0.16\\
99.85 \plm 0.20 & 100.00 \plm 0.00\\
\hline 
\end{tabular}

\begin{tabular}{|c|c|}
\multicolumn{2}{c}{tropic10} \\
\hline
\hline
OvO & OvA \\
\hline 95.66 \plm 1.16 & 96.40 \plm 0.88\\
97.53 \plm 0.56 & 98.71 \plm 0.51\\
99.30 \plm 0.45 & 99.80 \plm 0.34\\
99.49 \plm 0.30 & 99.80 \plm 0.34\\
99.57 \plm 0.32 & 99.80 \plm 0.20\\
\hline 
\end{tabular}

\begin{tabular}{|c|c|}
\multicolumn{2}{c}{tropic20} \\
\hline
\hline
OvO & OvA \\
\hline 92.76 \plm 1.15 & 95.38 \plm 0.64\\
95.83 \plm 0.52 & 98.01 \plm 0.24\\
98.39 \plm 0.34 & 99.18 \plm 0.22\\
99.28 \plm 0.20 & 99.60 \plm 0.12\\
99.41 \plm 0.26 & 99.75 \plm 0.16\\
\hline 
\end{tabular}
}%
\vspace{2mm}
\resizebox{\textwidth}{!}{
\begin{tabular}{|c|c|c|}
\multicolumn{3}{c}{swedishLeaves3folds3} \\
\hline
\hline
\scriptsize{Train Prozent} & OvO & OvA \\
\hline 10 & 88.44 \plm 3.29 & 90.00 \plm 1.15\\
20 & 93.56 \plm 2.14 & 93.11 \plm 1.92\\
50 & 96.44 \plm 2.52 & 97.11 \plm 3.01\\
80 & 99.78 \plm 0.38 & 98.67 \plm 1.15\\
100 & 99.78 \plm 0.38 & 99.11 \plm 0.38\\
\hline 
\end{tabular}

\begin{tabular}{|c|c|}
\multicolumn{2}{c}{swedishLeaves3folds5} \\
\hline
\hline
OvO & OvA \\
\hline 92.80 \plm 0.69 & 92.93 \plm 0.92\\
96.27 \plm 4.41 & 96.67 \plm 2.72\\
99.87 \plm 0.23 & 99.33 \plm 0.61\\
99.33 \plm 0.83 & 100.00 \plm 0.00\\
99.87 \plm 0.23 & 100.00 \plm 0.00\\
\hline 
\end{tabular}

\begin{tabular}{|c|c|}
\multicolumn{2}{c}{swedishLeaves3folds10} \\
\hline
\hline
OvO & OvA \\
\hline 97.73 \plm 0.12 & 95.33 \plm 1.17\\
98.53 \plm 0.31 & 97.13 \plm 0.50\\
98.47 \plm 1.03 & 99.33 \plm 0.58\\
99.80 \plm 0.20 & 99.93 \plm 0.12\\
99.80 \plm 0.00 & 99.80 \plm 0.20\\
\hline 
\end{tabular}

\begin{tabular}{|c|c|}
\multicolumn{2}{c}{swedishLeaves3folds15} \\
\hline
\hline
OvO & OvA \\
\hline 97.33 \plm 0.40 & 95.91 \plm 1.02\\
99.51 \plm 0.20 & 98.93 \plm 1.29\\
99.42 \plm 0.41 & 99.29 \plm 0.43\\
99.87 \plm 0.23 & 99.64 \plm 0.41\\
99.82 \plm 0.08 & 99.96 \plm 0.08\\
\hline 
\end{tabular}
}%
\vspace{2mm}
\resizebox{\textwidth}{!}{
\begin{tabular}{|c|c|c|}
\multicolumn{3}{c}{pawara-tropic10} \\
\hline
\hline
\scriptsize{Train Prozent} & OvO & OvA \\
\hline 10 & 93.90 \plm 0.91 & 95.72 \plm 0.88\\
20 & 97.08 \plm 1.25 & 98.02 \plm 0.46\\
50 & 99.22 \plm 0.21 & 99.66 \plm 0.24\\
80 & 99.45 \plm 0.11 & 99.74 \plm 0.24\\
100 & 99.45 \plm 0.25 & 99.92 \plm 0.12\\
\hline 
\end{tabular}

\begin{tabular}{|c|c|}
\multicolumn{2}{c}{pawara-monkey10} \\
\hline
\hline
OvO & OvA \\
\hline 97.08 \plm 0.68 & 96.71 \plm 0.94\\
97.23 \plm 1.33 & 97.81 \plm 0.93\\
98.25 \plm 0.55 & 98.32 \plm 1.11\\
98.40 \plm 0.95 & 98.54 \plm 0.25\\
98.62 \plm 0.70 & 99.20 \plm 0.31\\
\hline 
\end{tabular}

\begin{tabular}{|c|c|}
\multicolumn{2}{c}{pawara-umonkey10} \\
\hline
\hline
OvO & OvA \\
\hline 90.31 \plm 3.22 & 85.93 \plm 3.51\\
94.61 \plm 1.94 & 92.71 \plm 1.59\\
95.32 \plm 1.35 & 94.74 \plm 0.76\\
96.13 \plm 1.75 & 97.00 \plm 1.71\\
95.91 \plm 1.39 & 96.50 \plm 0.34\\
\hline 
\end{tabular}
}%

\end{figure}



\subsection{InceptionV3-Pawara Scratch}
\begin{figure}[H]
\resizebox{\textwidth}{!}{
\begin{tabular}{|c|c|c|}
\multicolumn{3}{c}{agrilplant3} \\
\hline
\hline
\scriptsize{Train Prozent} & OvO & OvA \\
\hline 10 & 81.33 \plm 2.24 & 81.22 \plm 3.54\\
20 & 87.67 \plm 3.37 & 88.22 \plm 3.54\\
50 & 97.56 \plm 0.63 & 97.33 \plm 0.61\\
80 & 98.56 \plm 0.63 & 98.00 \plm 1.39\\
100 & 98.00 \plm 0.93 & 98.33 \plm 0.79\\
\hline 
\end{tabular}

\begin{tabular}{|c|c|}
\multicolumn{2}{c}{agrilplant5} \\
\hline
\hline
OvO & OvA \\
\hline 78.40 \plm 3.43 & 81.47 \plm 2.48\\
89.47 \plm 2.39 & 89.53 \plm 3.90\\
97.53 \plm 0.38 & 97.27 \plm 0.76\\
98.00 \plm 0.71 & 98.27 \plm 0.28\\
98.33 \plm 0.62 & 98.87 \plm 0.61\\
\hline 
\end{tabular}

\begin{tabular}{|c|c|}
\multicolumn{2}{c}{agrilplant10} \\
\hline
\hline
OvO & OvA \\
\hline 75.83 \plm 3.31 & 75.90 \plm 2.99\\
86.60 \plm 1.39 & 87.87 \plm 0.69\\
93.80 \plm 0.95 & 95.07 \plm 0.49\\
96.17 \plm 0.70 & 96.37 \plm 0.36\\
96.77 \plm 0.35 & 97.27 \plm 0.32\\
\hline 
\end{tabular}
}%
\vspace{2mm}
\resizebox{\textwidth}{!}{
\begin{tabular}{|c|c|c|}
\multicolumn{3}{c}{tropic3} \\
\hline
\hline
\scriptsize{Train Prozent} & OvO & OvA \\
\hline 10 & 88.59 \plm 3.29 & 87.26 \plm 3.08\\
20 & 93.57 \plm 1.39 & 93.20 \plm 2.03\\
50 & 98.42 \plm 0.69 & 97.81 \plm 0.93\\
80 & 98.91 \plm 0.67 & 98.30 \plm 0.51\\
100 & 99.03 \plm 0.92 & 99.03 \plm 0.69\\
\hline 
\end{tabular}

\begin{tabular}{|c|c|}
\multicolumn{2}{c}{tropic5} \\
\hline
\hline
OvO & OvA \\
\hline 81.19 \plm 1.89 & 79.30 \plm 2.42\\
90.92 \plm 2.49 & 88.97 \plm 3.67\\
97.61 \plm 1.22 & 96.81 \plm 0.70\\
98.69 \plm 0.91 & 98.40 \plm 0.55\\
99.13 \plm 0.32 & 99.13 \plm 0.32\\
\hline 
\end{tabular}

\begin{tabular}{|c|c|}
\multicolumn{2}{c}{tropic10} \\
\hline
\hline
OvO & OvA \\
\hline 80.08 \plm 1.89 & 74.05 \plm 2.30\\
89.43 \plm 1.77 & 85.83 \plm 2.54\\
96.87 \plm 0.73 & 95.81 \plm 0.86\\
98.43 \plm 0.44 & 97.69 \plm 0.85\\
98.67 \plm 0.83 & 98.98 \plm 0.51\\
\hline 
\end{tabular}

\begin{tabular}{|c|c|}
\multicolumn{2}{c}{tropic20} \\
\hline
\hline
OvO & OvA \\
\hline 73.52 \plm 1.77 & 71.97 \plm 0.61\\
84.80 \plm 1.72 & 84.17 \plm 1.48\\
95.60 \plm 0.84 & 95.72 \plm 1.21\\
98.01 \plm 0.60 & 97.69 \plm 0.41\\
98.46 \plm 1.01 & 98.46 \plm 0.67\\
\hline 
\end{tabular}
}%
\vspace{2mm}
\resizebox{\textwidth}{!}{
\begin{tabular}{|c|c|c|}
\multicolumn{3}{c}{swedishLeaves3folds3} \\
\hline
\hline
\scriptsize{Train Prozent} & OvO & OvA \\
\hline 10 & 76.22 \plm 8.70 & 59.33 \plm 10.07\\
20 & 80.89 \plm 3.91 & 74.00 \plm 13.32\\
50 & 84.67 \plm 3.06 & 87.33 \plm 4.81\\
80 & 94.89 \plm 4.44 & 92.44 \plm 3.67\\
100 & 95.78 \plm 3.01 & 95.11 \plm 1.68\\
\hline 
\end{tabular}

\begin{tabular}{|c|c|}
\multicolumn{2}{c}{swedishLeaves3folds5} \\
\hline
\hline
OvO & OvA \\
\hline 68.93 \plm 16.11 & 65.20 \plm 8.38\\
85.20 \plm 0.69 & 84.53 \plm 4.64\\
93.20 \plm 1.60 & 91.20 \plm 1.20\\
96.13 \plm 2.95 & 96.27 \plm 0.23\\
96.53 \plm 2.31 & 95.60 \plm 2.43\\
\hline 
\end{tabular}

\begin{tabular}{|c|c|}
\multicolumn{2}{c}{swedishLeaves3folds10} \\
\hline
\hline
OvO & OvA \\
\hline 84.80 \plm 2.80 & 79.20 \plm 2.50\\
92.47 \plm 0.76 & 90.73 \plm 1.33\\
93.80 \plm 2.23 & 91.60 \plm 1.31\\
97.07 \plm 1.36 & 96.00 \plm 0.72\\
97.87 \plm 1.14 & 96.93 \plm 0.31\\
\hline 
\end{tabular}

\begin{tabular}{|c|c|}
\multicolumn{2}{c}{swedishLeaves3folds15} \\
\hline
\hline
OvO & OvA \\
\hline 84.53 \plm 4.19 & 76.53 \plm 4.19\\
92.71 \plm 0.56 & 90.71 \plm 2.21\\
96.22 \plm 0.73 & 92.44 \plm 2.83\\
97.96 \plm 0.47 & 98.09 \plm 0.66\\
98.00 \plm 0.80 & 98.49 \plm 0.73\\
\hline 
\end{tabular}
}%
\vspace{2mm}
\resizebox{\textwidth}{!}{
\begin{tabular}{|c|c|c|}
\multicolumn{3}{c}{pawara-tropic10} \\
\hline
\hline
\scriptsize{Train Prozent} & OvO & OvA \\
\hline 10 & 74.64 \plm 2.87 & 69.09 \plm 4.20\\
20 & 85.08 \plm 0.95 & 83.54 \plm 3.27\\
50 & 96.71 \plm 0.88 & 94.84 \plm 0.63\\
80 & 98.57 \plm 0.34 & 97.81 \plm 0.36\\
100 & 98.91 \plm 0.49 & 98.15 \plm 0.58\\
\hline 
\end{tabular}

\begin{tabular}{|c|c|}
\multicolumn{2}{c}{pawara-monkey10} \\
\hline
\hline
OvO & OvA \\
\hline 52.04 \plm 2.57 & 50.76 \plm 5.07\\
66.94 \plm 2.07 & 63.20 \plm 2.95\\
81.97 \plm 2.21 & 81.23 \plm 1.57\\
89.77 \plm 1.99 & 89.63 \plm 1.79\\
92.33 \plm 2.28 & 92.78 \plm 1.26\\
\hline 
\end{tabular}

\begin{tabular}{|c|c|}
\multicolumn{2}{c}{pawara-umonkey10} \\
\hline
\hline
OvO & OvA \\
\hline 39.14 \plm 2.83 & 37.37 \plm 2.38\\
48.99 \plm 2.75 & 45.49 \plm 5.20\\
62.18 \plm 4.63 & 63.06 \plm 4.26\\
72.70 \plm 2.05 & 70.66 \plm 2.04\\
73.21 \plm 2.05 & 74.96 \plm 3.24\\
\hline 
\end{tabular}
}%

\end{figure}

\subsection{InceptionV3-Pawara Finetune}
\begin{figure}[H]
\resizebox{\textwidth}{!}{
\begin{tabular}{|c|c|c|}
\multicolumn{3}{c}{agrilplant3} \\
\hline
\hline
\scriptsize{Train Prozent} & OvO & OvA \\
\hline 10 & 99.33 \plm 0.61 & 99.44 \plm 0.39\\
20 & 99.78 \plm 0.30 & 99.33 \plm 0.91\\
50 & 99.89 \plm 0.25 & 100.00 \plm 0.00\\
80 & 100.00 \plm 0.00 & 100.00 \plm 0.00\\
100 & 100.00 \plm 0.00 & 100.00 \plm 0.00\\
\hline 
\end{tabular}

\begin{tabular}{|c|c|}
\multicolumn{2}{c}{agrilplant5} \\
\hline
\hline
OvO & OvA \\
\hline 95.73 \plm 1.21 & 96.00 \plm 1.80\\
97.13 \plm 0.96 & 97.47 \plm 0.87\\
99.20 \plm 0.56 & 98.93 \plm 0.43\\
99.53 \plm 0.45 & 99.67 \plm 0.58\\
99.67 \plm 0.24 & 99.80 \plm 0.30\\
\hline 
\end{tabular}

\begin{tabular}{|c|c|}
\multicolumn{2}{c}{agrilplant10} \\
\hline
\hline
OvO & OvA \\
\hline 92.20 \plm 1.54 & 94.03 \plm 1.24\\
94.73 \plm 0.69 & 96.03 \plm 0.76\\
97.33 \plm 0.73 & 98.07 \plm 0.53\\
97.77 \plm 0.61 & 98.67 \plm 0.39\\
98.13 \plm 0.56 & 98.90 \plm 0.60\\
\hline 
\end{tabular}
}%
\vspace{2mm}
\resizebox{\textwidth}{!}{
\begin{tabular}{|c|c|c|}
\multicolumn{3}{c}{tropic3} \\
\hline
\hline
\scriptsize{Train Prozent} & OvO & OvA \\
\hline 10 & 98.79 \plm 1.29 & 98.67 \plm 1.08\\
20 & 99.76 \plm 0.33 & 99.76 \plm 0.33\\
50 & 99.88 \plm 0.27 & 99.76 \plm 0.54\\
80 & 99.88 \plm 0.27 & 99.64 \plm 0.54\\
100 & 99.88 \plm 0.27 & 99.76 \plm 0.33\\
\hline 
\end{tabular}

\begin{tabular}{|c|c|}
\multicolumn{2}{c}{tropic5} \\
\hline
\hline
OvO & OvA \\
\hline 97.38 \plm 1.04 & 97.17 \plm 0.91\\
99.27 \plm 0.51 & 99.06 \plm 0.32\\
99.35 \plm 0.60 & 99.71 \plm 0.30\\
99.64 \plm 0.44 & 99.64 \plm 0.36\\
99.93 \plm 0.16 & 99.93 \plm 0.16\\
\hline 
\end{tabular}

\begin{tabular}{|c|c|}
\multicolumn{2}{c}{tropic10} \\
\hline
\hline
OvO & OvA \\
\hline 93.46 \plm 2.07 & 96.48 \plm 0.95\\
97.06 \plm 0.44 & 97.81 \plm 1.19\\
99.06 \plm 0.45 & 99.61 \plm 0.42\\
99.37 \plm 0.49 & 99.84 \plm 0.26\\
99.45 \plm 0.47 & 99.88 \plm 0.11\\
\hline 
\end{tabular}

\begin{tabular}{|c|c|}
\multicolumn{2}{c}{tropic20} \\
\hline
\hline
OvO & OvA \\
\hline 93.76 \plm 1.54 & 95.49 \plm 0.63\\
87.15 \plm 18.49 & 97.71 \plm 0.41\\
97.97 \plm 0.27 & 99.18 \plm 0.22\\
99.11 \plm 0.56 & 99.60 \plm 0.29\\
99.37 \plm 0.17 & 99.66 \plm 0.18\\
\hline 
\end{tabular}
}%
\vspace{2mm}
\resizebox{\textwidth}{!}{
\begin{tabular}{|c|c|c|}
\multicolumn{3}{c}{swedishLeaves3folds3} \\
\hline
\hline
\scriptsize{Train Prozent} & OvO & OvA \\
\hline 10 & 89.11 \plm 5.18 & 88.89 \plm 3.91\\
20 & 92.67 \plm 6.93 & 94.44 \plm 3.42\\
50 & 93.78 \plm 5.55 & 94.22 \plm 3.29\\
80 & 99.56 \plm 0.38 & 99.78 \plm 0.38\\
100 & 99.78 \plm 0.38 & 100.00 \plm 0.00\\
\hline 
\end{tabular}

\begin{tabular}{|c|c|}
\multicolumn{2}{c}{swedishLeaves3folds5} \\
\hline
\hline
OvO & OvA \\
\hline 92.53 \plm 1.62 & 89.20 \plm 1.60\\
96.40 \plm 2.62 & 97.20 \plm 2.08\\
99.87 \plm 0.23 & 99.60 \plm 0.40\\
99.87 \plm 0.23 & 100.00 \plm 0.00\\
99.87 \plm 0.23 & 99.87 \plm 0.23\\
\hline 
\end{tabular}

\begin{tabular}{|c|c|}
\multicolumn{2}{c}{swedishLeaves3folds10} \\
\hline
\hline
OvO & OvA \\
\hline 96.27 \plm 1.86 & 95.00 \plm 0.53\\
97.93 \plm 0.42 & 97.33 \plm 0.92\\
98.60 \plm 0.53 & 98.93 \plm 0.81\\
99.93 \plm 0.12 & 99.87 \plm 0.12\\
99.80 \plm 0.20 & 99.87 \plm 0.23\\
\hline 
\end{tabular}

\begin{tabular}{|c|c|}
\multicolumn{2}{c}{swedishLeaves3folds15} \\
\hline
\hline
OvO & OvA \\
\hline 95.64 \plm 1.44 & 96.09 \plm 0.73\\
99.24 \plm 0.54 & 99.07 \plm 0.58\\
99.33 \plm 0.46 & 99.20 \plm 0.35\\
99.78 \plm 0.15 & 99.82 \plm 0.08\\
99.82 \plm 0.08 & 99.82 \plm 0.08\\
\hline 
\end{tabular}
}%
\vspace{2mm}
\resizebox{\textwidth}{!}{
\begin{tabular}{|c|c|c|}
\multicolumn{3}{c}{pawara-tropic10} \\
\hline
\hline
\scriptsize{Train Prozent} & OvO & OvA \\
\hline 10 & 93.58 \plm 1.27 & 94.81 \plm 0.80\\
20 & 96.14 \plm 1.51 & 97.86 \plm 0.61\\
50 & 98.64 \plm 0.65 & 99.63 \plm 0.30\\
80 & 99.37 \plm 0.11 & 99.71 \plm 0.25\\
100 & 99.48 \plm 0.16 & 99.84 \plm 0.11\\
\hline 
\end{tabular}

\begin{tabular}{|c|c|}
\multicolumn{2}{c}{pawara-monkey10} \\
\hline
\hline
OvO & OvA \\
\hline 95.47 \plm 1.80 & 96.93 \plm 1.77\\
96.93 \plm 1.38 & 98.10 \plm 0.95\\
97.37 \plm 1.15 & 98.10 \plm 0.61\\
97.59 \plm 1.05 & 99.20 \plm 0.65\\
99.05 \plm 0.55 & 99.42 \plm 0.33\\
\hline 
\end{tabular}

\begin{tabular}{|c|c|}
\multicolumn{2}{c}{pawara-umonkey10} \\
\hline
\hline
OvO & OvA \\
\hline 83.75 \plm 6.30 & 82.51 \plm 4.49\\
91.75 \plm 2.80 & 90.81 \plm 1.73\\
94.59 \plm 2.08 & 95.47 \plm 1.29\\
95.47 \plm 0.96 & 96.79 \plm 1.68\\
95.03 \plm 1.20 & 97.26 \plm 1.37\\
\hline 
\end{tabular}
}%

\end{figure}











\section{Tensorflow 2.4.1}

\subsection{ResNet-50 Scratch}
\begin{figure}[H]
\resizebox{\textwidth}{!}{
\begin{tabular}{|c|c|c|}
\multicolumn{3}{c}{agrilplant3} \\
\hline
\hline
\scriptsize{Train Prozent} & OvO & OvA \\
\hline 10 & 78.56 \plm 1.95 & 78.44 \plm 2.40\\
20 & 81.78 \plm 3.41 & 83.78 \plm 2.53\\
50 & 94.44 \plm 1.88 & 94.89 \plm 0.72\\
80 & 97.56 \plm 1.01 & 96.67 \plm 1.30\\
100 & 98.00 \plm 1.01 & 98.11 \plm 0.63\\
\hline 
\end{tabular}

\begin{tabular}{|c|c|}
\multicolumn{2}{c}{agrilplant5} \\
\hline
\hline
OvO & OvA \\
\hline 75.47 \plm 4.11 & 76.33 \plm 3.60\\
86.67 \plm 2.36 & 87.40 \plm 2.76\\
95.73 \plm 1.52 & 95.53 \plm 2.18\\
97.60 \plm 1.01 & 97.13 \plm 0.84\\
98.33 \plm 0.71 & 98.27 \plm 0.83\\
\hline 
\end{tabular}

\begin{tabular}{|c|c|}
\multicolumn{2}{c}{agrilplant10} \\
\hline
\hline
OvO & OvA \\
\hline 72.20 \plm 4.21 & 72.17 \plm 2.99\\
84.77 \plm 2.93 & 85.20 \plm 2.27\\
92.73 \plm 1.96 & 93.30 \plm 0.79\\
94.20 \plm 0.45 & 95.20 \plm 0.92\\
95.13 \plm 0.68 & 95.70 \plm 0.75\\
\hline 
\end{tabular}
}%
\vspace{2mm}
\resizebox{\textwidth}{!}{
\begin{tabular}{|c|c|c|}
\multicolumn{3}{c}{tropic3} \\
\hline
\hline
\scriptsize{Train Prozent} & OvO & OvA \\
\hline 10 & 82.78 \plm 10.61 & 86.78 \plm 3.80\\
20 & 93.32 \plm 2.54 & 92.84 \plm 1.78\\
50 & 97.57 \plm 1.36 & 96.96 \plm 1.83\\
80 & 98.06 \plm 1.00 & 97.82 \plm 1.10\\
100 & 99.03 \plm 0.92 & 98.91 \plm 0.90\\
\hline 
\end{tabular}

\begin{tabular}{|c|c|}
\multicolumn{2}{c}{tropic5} \\
\hline
\hline
OvO & OvA \\
\hline 75.45 \plm 3.15 & 78.50 \plm 3.59\\
88.02 \plm 3.46 & 87.51 \plm 2.49\\
96.73 \plm 1.45 & 96.01 \plm 1.35\\
98.26 \plm 0.79 & 97.68 \plm 1.11\\
98.77 \plm 0.75 & 98.69 \plm 0.87\\
\hline 
\end{tabular}

\begin{tabular}{|c|c|}
\multicolumn{2}{c}{tropic10} \\
\hline
\hline
OvO & OvA \\
\hline 74.44 \plm 3.17 & 72.68 \plm 2.09\\
83.68 \plm 4.03 & 84.34 \plm 1.01\\
94.01 \plm 1.17 & 95.34 \plm 0.47\\
96.52 \plm 0.97 & 97.46 \plm 0.83\\
97.69 \plm 0.64 & 98.00 \plm 0.51\\
\hline 
\end{tabular}

\begin{tabular}{|c|c|}
\multicolumn{2}{c}{tropic20} \\
\hline
\hline
OvO & OvA \\
\hline 68.73 \plm 3.76 & 71.23 \plm 3.21\\
79.76 \plm 3.52 & 84.57 \plm 1.14\\
91.89 \plm 1.27 & 94.45 \plm 0.35\\
95.30 \plm 1.10 & 96.89 \plm 0.55\\
96.72 \plm 0.64 & 97.71 \plm 0.41\\
\hline 
\end{tabular}
}%
\vspace{2mm}
\resizebox{\textwidth}{!}{
\begin{tabular}{|c|c|c|}
\multicolumn{3}{c}{swedishLeaves3folds3} \\
\hline
\hline
\scriptsize{Train Prozent} & OvO & OvA \\
\hline 10 & 33.33 \plm 0.00 & 33.33 \plm 0.00\\
20 & 33.33 \plm 0.00 & 32.89 \plm 0.77\\
50 & 43.11 \plm 16.94 & 40.67 \plm 6.43\\
80 & 67.33 \plm 11.02 & 64.22 \plm 22.19\\
100 & 93.56 \plm 5.98 & 95.78 \plm 0.38\\
\hline 
\end{tabular}

\begin{tabular}{|c|c|}
\multicolumn{2}{c}{swedishLeaves3folds5} \\
\hline
\hline
OvO & OvA \\
\hline 20.27 \plm 0.46 & 20.00 \plm 0.00\\
18.53 \plm 2.54 & 20.00 \plm 0.00\\
84.93 \plm 10.51 & 84.27 \plm 5.80\\
95.73 \plm 1.51 & 93.60 \plm 1.44\\
96.93 \plm 2.60 & 96.13 \plm 1.01\\
\hline 
\end{tabular}

\begin{tabular}{|c|c|}
\multicolumn{2}{c}{swedishLeaves3folds10} \\
\hline
\hline
OvO & OvA \\
\hline 10.07 \plm 0.12 & 13.07 \plm 5.31\\
83.80 \plm 3.08 & 79.13 \plm 2.32\\
91.80 \plm 1.04 & 89.60 \plm 0.53\\
95.20 \plm 0.20 & 92.47 \plm 0.95\\
96.33 \plm 1.03 & 96.07 \plm 1.36\\
\hline 
\end{tabular}

\begin{tabular}{|c|c|}
\multicolumn{2}{c}{swedishLeaves3folds15} \\
\hline
\hline
OvO & OvA \\
\hline 13.38 \plm 6.53 & 9.78 \plm 2.28\\
88.00 \plm 0.58 & 88.58 \plm 2.90\\
92.89 \plm 1.53 & 91.82 \plm 2.00\\
96.62 \plm 0.20 & 95.91 \plm 0.15\\
96.53 \plm 0.81 & 96.80 \plm 0.23\\
\hline 
\end{tabular}
}%
\vspace{2mm}
\resizebox{\textwidth}{!}{
\begin{tabular}{|c|c|c|}
\multicolumn{3}{c}{pawara-tropic10} \\
\hline
\hline
\scriptsize{Train Prozent} & OvO & OvA \\
\hline 10 & 68.88 \plm 3.93 & 66.53 \plm 3.45\\
20 & 82.29 \plm 1.06 & 81.79 \plm 2.51\\
50 & 93.45 \plm 0.91 & 93.66 \plm 1.22\\
80 & 96.24 \plm 0.69 & 96.45 \plm 0.75\\
100 & 97.03 \plm 1.39 & 97.47 \plm 0.50\\
\hline 
\end{tabular}

\begin{tabular}{|c|c|}
\multicolumn{2}{c}{pawara-monkey10} \\
\hline
\hline
OvO & OvA \\
\hline 43.88 \plm 4.22 & 45.56 \plm 5.51\\
57.60 \plm 4.25 & 60.06 \plm 1.80\\
70.29 \plm 9.51 & 75.32 \plm 2.58\\
84.96 \plm 3.30 & 85.69 \plm 1.61\\
86.36 \plm 3.09 & 89.35 \plm 1.62\\
\hline 
\end{tabular}

\begin{tabular}{|c|c|}
\multicolumn{2}{c}{pawara-umonkey10} \\
\hline
\hline
OvO & OvA \\
\hline 30.74 \plm 2.73 & 32.78 \plm 2.95\\
39.85 \plm 3.56 & 40.00 \plm 1.67\\
54.23 \plm 4.28 & 56.57 \plm 5.11\\
61.22 \plm 9.05 & 67.60 \plm 3.23\\
66.94 \plm 3.05 & 69.63 \plm 2.12\\
\hline 
\end{tabular}

\begin{tabular}{|c|c|}
\multicolumn{2}{c}{cifar10} \\
\hline
\hline
OvO & OvA \\
\hline 74.88 \plm 3.02 & 79.15 \plm 0.72\\
83.53 \plm 0.45 & 85.50 \plm 0.38\\
88.78 \plm 1.11 & 90.87 \plm 0.52\\
91.32 \plm 1.12 & 93.02 \plm 0.19\\
92.40 \plm 0.39 & 93.64 \plm 0.16\\
\hline 
\end{tabular}
}%

\end{figure}

\subsection{ResNet-50 Finetune}
\begin{figure}[H]
\resizebox{\textwidth}{!}{
\begin{tabular}{|c|c|c|}
\multicolumn{3}{c}{agrilplant3} \\
\hline
\hline
\scriptsize{Train Prozent} & OvO & OvA \\
\hline 10 & 35.22 \plm 2.17 & 33.78 \plm 0.99\\
20 & 93.00 \plm 3.01 & 95.00 \plm 2.12\\
50 & 99.78 \plm 0.30 & 99.67 \plm 0.50\\
80 & 100.00 \plm 0.00 & 99.89 \plm 0.25\\
100 & 100.00 \plm 0.00 & 100.00 \plm 0.00\\
\hline 
\end{tabular}

\begin{tabular}{|c|c|}
\multicolumn{2}{c}{agrilplant5} \\
\hline
\hline
OvO & OvA \\
\hline 57.87 \plm 9.53 & 52.13 \plm 3.69\\
98.47 \plm 0.61 & 98.27 \plm 1.36\\
99.00 \plm 0.62 & 99.07 \plm 0.64\\
99.67 \plm 0.24 & 99.60 \plm 0.37\\
99.60 \plm 0.15 & 99.73 \plm 0.37\\
\hline 
\end{tabular}

\begin{tabular}{|c|c|}
\multicolumn{2}{c}{agrilplant10} \\
\hline
\hline
OvO & OvA \\
\hline 91.97 \plm 0.72 & 93.07 \plm 1.07\\
95.17 \plm 0.51 & 95.90 \plm 0.25\\
97.03 \plm 0.38 & 98.10 \plm 0.55\\
98.03 \plm 0.36 & 98.20 \plm 0.36\\
97.83 \plm 0.31 & 98.53 \plm 0.41\\
\hline 
\end{tabular}
}%
\vspace{2mm}
\resizebox{\textwidth}{!}{
\begin{tabular}{|c|c|c|}
\multicolumn{3}{c}{tropic3} \\
\hline
\hline
\scriptsize{Train Prozent} & OvO & OvA \\
\hline 10 & 37.73 \plm 5.65 & 33.99 \plm 7.04\\
20 & 81.19 \plm 7.81 & 81.20 \plm 11.39\\
50 & 99.39 \plm 0.43 & 99.51 \plm 0.51\\
80 & 99.76 \plm 0.33 & 99.76 \plm 0.33\\
100 & 99.76 \plm 0.33 & 99.88 \plm 0.27\\
\hline 
\end{tabular}

\begin{tabular}{|c|c|}
\multicolumn{2}{c}{tropic5} \\
\hline
\hline
OvO & OvA \\
\hline 48.96 \plm 8.84 & 36.17 \plm 8.61\\
98.33 \plm 0.55 & 98.91 \plm 0.51\\
99.27 \plm 0.26 & 99.56 \plm 0.40\\
99.64 \plm 0.36 & 99.42 \plm 0.55\\
99.85 \plm 0.20 & 99.85 \plm 0.33\\
\hline 
\end{tabular}

\begin{tabular}{|c|c|}
\multicolumn{2}{c}{tropic10} \\
\hline
\hline
OvO & OvA \\
\hline 92.84 \plm 2.31 & 94.76 \plm 1.65\\
96.36 \plm 0.45 & 97.65 \plm 0.79\\
98.98 \plm 0.53 & 99.53 \plm 0.30\\
99.26 \plm 0.26 & 99.73 \plm 0.18\\
99.33 \plm 0.43 & 99.88 \plm 0.26\\
\hline 
\end{tabular}

\begin{tabular}{|c|c|}
\multicolumn{2}{c}{tropic20} \\
\hline
\hline
OvO & OvA \\
\hline 91.70 \plm 0.86 & 93.92 \plm 0.48\\
95.81 \plm 0.65 & 97.33 \plm 0.36\\
98.22 \plm 0.39 & 99.00 \plm 0.45\\
98.90 \plm 0.16 & 99.37 \plm 0.41\\
99.28 \plm 0.13 & 99.62 \plm 0.27\\
\hline 
\end{tabular}
}%
\vspace{2mm}
\resizebox{\textwidth}{!}{
\begin{tabular}{|c|c|c|}
\multicolumn{3}{c}{swedishLeaves3folds3} \\
\hline
\hline
\scriptsize{Train Prozent} & OvO & OvA \\
\hline 10 & 33.33 \plm 0.00 & 25.11 \plm 14.24\\
20 & 33.33 \plm 0.00 & 33.56 \plm 0.38\\
50 & 37.11 \plm 6.54 & 33.33 \plm 0.00\\
80 & 33.78 \plm 16.67 & 41.33 \plm 20.99\\
100 & 50.67 \plm 13.38 & 45.78 \plm 18.20\\
\hline 
\end{tabular}

\begin{tabular}{|c|c|}
\multicolumn{2}{c}{swedishLeaves3folds5} \\
\hline
\hline
OvO & OvA \\
\hline 20.00 \plm 0.00 & 19.73 \plm 1.22\\
20.00 \plm 0.00 & 20.00 \plm 0.00\\
29.60 \plm 16.98 & 16.00 \plm 4.61\\
50.27 \plm 19.31 & 67.33 \plm 11.90\\
60.00 \plm 10.89 & 79.07 \plm 2.20\\
\hline 
\end{tabular}

\begin{tabular}{|c|c|}
\multicolumn{2}{c}{swedishLeaves3folds10} \\
\hline
\hline
OvO & OvA \\
\hline 12.87 \plm 3.42 & 10.07 \plm 0.12\\
11.27 \plm 4.24 & 12.00 \plm 3.46\\
89.20 \plm 3.22 & 80.67 \plm 9.65\\
99.40 \plm 0.53 & 99.60 \plm 0.53\\
99.67 \plm 0.23 & 99.87 \plm 0.12\\
\hline 
\end{tabular}

\begin{tabular}{|c|c|}
\multicolumn{2}{c}{swedishLeaves3folds15} \\
\hline
\hline
OvO & OvA \\
\hline 6.09 \plm 1.12 & 7.69 \plm 3.64\\
13.96 \plm 8.96 & 14.00 \plm 3.43\\
99.11 \plm 0.62 & 99.16 \plm 0.34\\
99.64 \plm 0.28 & 99.60 \plm 0.48\\
99.64 \plm 0.08 & 99.64 \plm 0.20\\
\hline 
\end{tabular}
}%
\vspace{2mm}
\resizebox{\textwidth}{!}{
\begin{tabular}{|c|c|c|}
\multicolumn{3}{c}{pawara-tropic10} \\
\hline
\hline
\scriptsize{Train Prozent} & OvO & OvA \\
\hline 10 & 88.65 \plm 1.18 & 88.60 \plm 2.22\\
20 & 95.72 \plm 1.74 & 97.21 \plm 0.47\\
50 & 98.46 \plm 0.52 & 99.45 \plm 0.25\\
80 & 99.19 \plm 0.19 & 99.69 \plm 0.27\\
100 & 99.40 \plm 0.22 & 99.66 \plm 0.12\\
\hline 
\end{tabular}

\begin{tabular}{|c|c|}
\multicolumn{2}{c}{pawara-monkey10} \\
\hline
\hline
OvO & OvA \\
\hline 14.39 \plm 1.98 & 14.02 \plm 1.91\\
93.28 \plm 2.05 & 93.13 \plm 2.34\\
94.82 \plm 0.49 & 95.90 \plm 1.88\\
96.93 \plm 1.78 & 97.08 \plm 0.94\\
96.86 \plm 1.13 & 97.52 \plm 0.46\\
\hline 
\end{tabular}

\begin{tabular}{|c|c|}
\multicolumn{2}{c}{pawara-umonkey10} \\
\hline
\hline
OvO & OvA \\
\hline 10.51 \plm 1.60 & 8.98 \plm 1.16\\
21.29 \plm 10.29 & 21.63 \plm 7.45\\
91.31 \plm 1.82 & 88.24 \plm 2.88\\
91.75 \plm 1.52 & 89.70 \plm 2.00\\
91.75 \plm 1.62 & 92.48 \plm 0.96\\
\hline 
\end{tabular}
}%

\end{figure}



\subsection{InceptionV3 Scratch}
\begin{figure}[H]
\resizebox{\textwidth}{!}{
\begin{tabular}{|c|c|c|}
\multicolumn{3}{c}{agrilplant3} \\
\hline
\hline
\scriptsize{Train Prozent} & OvO & OvA \\
\hline 10 & 82.56 \plm 3.03 & 83.22 \plm 4.50\\
20 & 86.11 \plm 2.48 & 87.33 \plm 2.34\\
50 & 97.11 \plm 0.46 & 97.00 \plm 0.63\\
80 & 98.22 \plm 1.44 & 98.22 \plm 1.27\\
100 & 98.44 \plm 0.61 & 97.89 \plm 0.91\\
\hline 
\end{tabular}

\begin{tabular}{|c|c|}
\multicolumn{2}{c}{agrilplant5} \\
\hline
\hline
OvO & OvA \\
\hline 82.33 \plm 3.82 & 80.13 \plm 3.61\\
91.13 \plm 1.95 & 89.60 \plm 3.85\\
97.67 \plm 1.05 & 97.20 \plm 0.61\\
98.40 \plm 0.68 & 98.73 \plm 0.43\\
98.67 \plm 0.47 & 98.73 \plm 0.95\\
\hline 
\end{tabular}

\begin{tabular}{|c|c|}
\multicolumn{2}{c}{agrilplant10} \\
\hline
\hline
OvO & OvA \\
\hline 78.57 \plm 1.42 & 74.53 \plm 3.26\\
86.87 \plm 0.57 & 85.63 \plm 1.50\\
94.10 \plm 0.61 & 93.70 \plm 1.14\\
96.33 \plm 0.44 & 96.33 \plm 0.63\\
96.50 \plm 0.82 & 97.50 \plm 0.58\\
\hline 
\end{tabular}
}%
\vspace{2mm}
\resizebox{\textwidth}{!}{
\begin{tabular}{|c|c|c|}
\multicolumn{3}{c}{tropic3} \\
\hline
\hline
\scriptsize{Train Prozent} & OvO & OvA \\
\hline 10 & 88.59 \plm 3.10 & 89.56 \plm 3.30\\
20 & 94.18 \plm 2.25 & 92.11 \plm 2.38\\
50 & 97.82 \plm 1.40 & 97.21 \plm 1.96\\
80 & 98.79 \plm 0.96 & 98.79 \plm 1.29\\
100 & 98.79 \plm 0.74 & 98.91 \plm 0.51\\
\hline 
\end{tabular}

\begin{tabular}{|c|c|}
\multicolumn{2}{c}{tropic5} \\
\hline
\hline
OvO & OvA \\
\hline 81.41 \plm 3.52 & 76.90 \plm 4.75\\
89.04 \plm 1.50 & 87.95 \plm 3.88\\
97.24 \plm 0.83 & 97.10 \plm 1.06\\
98.98 \plm 0.79 & 98.40 \plm 0.55\\
99.20 \plm 0.70 & 98.91 \plm 0.57\\
\hline 
\end{tabular}

\begin{tabular}{|c|c|}
\multicolumn{2}{c}{tropic10} \\
\hline
\hline
OvO & OvA \\
\hline 77.14 \plm 2.65 & 72.33 \plm 2.96\\
88.61 \plm 1.83 & 83.13 \plm 2.63\\
97.53 \plm 0.66 & 96.09 \plm 1.27\\
98.40 \plm 0.93 & 98.08 \plm 0.80\\
99.26 \plm 0.59 & 98.79 \plm 0.80\\
\hline 
\end{tabular}

\begin{tabular}{|c|c|}
\multicolumn{2}{c}{tropic20} \\
\hline
\hline
OvO & OvA \\
\hline 73.60 \plm 2.20 & 68.61 \plm 1.19\\
85.92 \plm 1.99 & 83.72 \plm 1.51\\
96.06 \plm 0.84 & 95.20 \plm 0.62\\
97.71 \plm 0.87 & 96.89 \plm 0.60\\
98.71 \plm 0.29 & 98.43 \plm 0.16\\
\hline 
\end{tabular}
}%
\vspace{2mm}
\resizebox{\textwidth}{!}{
\begin{tabular}{|c|c|c|}
\multicolumn{3}{c}{swedishLeaves3folds3} \\
\hline
\hline
\scriptsize{Train Prozent} & OvO & OvA \\
\hline 10 & 33.33 \plm 0.00 & 33.33 \plm 0.00\\
20 & 33.33 \plm 0.00 & 34.44 \plm 1.92\\
50 & 37.78 \plm 3.91 & 41.78 \plm 7.34\\
80 & 83.11 \plm 9.25 & 73.56 \plm 25.10\\
100 & 94.67 \plm 4.67 & 95.33 \plm 1.76\\
\hline 
\end{tabular}

\begin{tabular}{|c|c|}
\multicolumn{2}{c}{swedishLeaves3folds5} \\
\hline
\hline
OvO & OvA \\
\hline 20.13 \plm 0.23 & 20.00 \plm 0.00\\
20.00 \plm 0.00 & 20.00 \plm 0.00\\
86.13 \plm 5.03 & 89.33 \plm 5.21\\
96.80 \plm 0.40 & 94.80 \plm 1.06\\
98.00 \plm 0.40 & 97.73 \plm 1.85\\
\hline 
\end{tabular}

\begin{tabular}{|c|c|}
\multicolumn{2}{c}{swedishLeaves3folds10} \\
\hline
\hline
OvO & OvA \\
\hline 9.73 \plm 0.46 & 13.07 \plm 5.31\\
90.20 \plm 2.80 & 79.07 \plm 3.52\\
93.27 \plm 1.22 & 92.73 \plm 1.67\\
97.53 \plm 1.70 & 96.47 \plm 0.95\\
97.67 \plm 0.31 & 98.20 \plm 0.20\\
\hline 
\end{tabular}

\begin{tabular}{|c|c|}
\multicolumn{2}{c}{swedishLeaves3folds15} \\
\hline
\hline
OvO & OvA \\
\hline 11.69 \plm 4.82 & 13.42 \plm 5.92\\
92.53 \plm 3.01 & 91.69 \plm 1.55\\
96.40 \plm 1.76 & 92.80 \plm 1.04\\
97.47 \plm 0.53 & 97.02 \plm 0.20\\
98.36 \plm 0.63 & 98.09 \plm 0.67\\
\hline 
\end{tabular}
}%
\vspace{2mm}
\resizebox{\textwidth}{!}{
\begin{tabular}{|c|c|c|}
\multicolumn{3}{c}{pawara-tropic10} \\
\hline
\hline
\scriptsize{Train Prozent} & OvO & OvA \\
\hline 10 & 74.18 \plm 3.94 & 66.34 \plm 3.70\\
20 & 86.12 \plm 2.04 & 81.69 \plm 1.95\\
50 & 96.50 \plm 0.64 & 94.26 \plm 1.38\\
80 & 98.31 \plm 0.63 & 97.10 \plm 0.60\\
100 & 98.75 \plm 0.37 & 98.25 \plm 0.48\\
\hline 
\end{tabular}

\begin{tabular}{|c|c|}
\multicolumn{2}{c}{pawara-monkey10} \\
\hline
\hline
OvO & OvA \\
\hline 49.56 \plm 1.15 & 50.39 \plm 5.07\\
65.83 \plm 2.11 & 62.98 \plm 2.25\\
78.10 \plm 2.31 & 78.39 \plm 1.42\\
89.56 \plm 2.31 & 89.49 \plm 2.60\\
92.34 \plm 1.78 & 92.26 \plm 2.11\\
\hline 
\end{tabular}

\begin{tabular}{|c|c|}
\multicolumn{2}{c}{pawara-umonkey10} \\
\hline
\hline
OvO & OvA \\
\hline 36.21 \plm 3.97 & 37.01 \plm 2.05\\
47.22 \plm 5.86 & 44.60 \plm 3.79\\
62.18 \plm 4.31 & 59.71 \plm 3.00\\
70.16 \plm 2.21 & 70.23 \plm 2.37\\
75.40 \plm 2.61 & 74.40 \plm 4.59\\
\hline 
\end{tabular}
}%

\end{figure}
\subsection{InceptionV3 Finetune}
\begin{figure}[H]
\resizebox{\textwidth}{!}{
\begin{tabular}{|c|c|c|}
\multicolumn{3}{c}{agrilplant3} \\
\hline
\hline
\scriptsize{Train Prozent} & OvO & OvA \\
\hline 10 & 99.56 \plm 0.25 & 99.56 \plm 0.46\\
20 & 99.78 \plm 0.50 & 99.44 \plm 0.39\\
50 & 99.89 \plm 0.25 & 99.89 \plm 0.25\\
80 & 99.78 \plm 0.30 & 100.00 \plm 0.00\\
100 & 99.89 \plm 0.25 & 100.00 \plm 0.00\\
\hline 
\end{tabular}

\begin{tabular}{|c|c|}
\multicolumn{2}{c}{agrilplant5} \\
\hline
\hline
OvO & OvA \\
\hline 96.27 \plm 2.07 & 96.27 \plm 1.53\\
97.53 \plm 0.90 & 98.67 \plm 0.78\\
99.20 \plm 0.45 & 99.00 \plm 0.33\\
99.60 \plm 0.37 & 99.87 \plm 0.18\\
99.47 \plm 0.30 & 99.80 \plm 0.18\\
\hline 
\end{tabular}

\begin{tabular}{|c|c|}
\multicolumn{2}{c}{agrilplant10} \\
\hline
\hline
OvO & OvA \\
\hline 93.43 \plm 0.76 & 94.13 \plm 0.57\\
95.63 \plm 0.68 & 96.50 \plm 1.02\\
97.37 \plm 0.64 & 98.30 \plm 0.80\\
98.27 \plm 0.51 & 98.57 \plm 0.56\\
98.20 \plm 0.61 & 98.87 \plm 0.32\\
\hline 
\end{tabular}
}%
\vspace{2mm}
\resizebox{\textwidth}{!}{
\begin{tabular}{|c|c|c|}
\multicolumn{3}{c}{tropic3} \\
\hline
\hline
\scriptsize{Train Prozent} & OvO & OvA \\
\hline 10 & 98.79 \plm 1.13 & 97.58 \plm 2.01\\
20 & 99.52 \plm 0.79 & 99.76 \plm 0.33\\
50 & 99.76 \plm 0.33 & 99.76 \plm 0.33\\
80 & 99.88 \plm 0.27 & 99.51 \plm 0.51\\
100 & 100.00 \plm 0.00 & 99.76 \plm 0.54\\
\hline 
\end{tabular}

\begin{tabular}{|c|c|}
\multicolumn{2}{c}{tropic5} \\
\hline
\hline
OvO & OvA \\
\hline 97.53 \plm 1.01 & 97.10 \plm 0.85\\
99.06 \plm 0.55 & 99.06 \plm 0.71\\
99.42 \plm 0.55 & 99.64 \plm 0.26\\
99.71 \plm 0.30 & 99.93 \plm 0.16\\
99.85 \plm 0.20 & 99.85 \plm 0.20\\
\hline 
\end{tabular}

\begin{tabular}{|c|c|}
\multicolumn{2}{c}{tropic10} \\
\hline
\hline
OvO & OvA \\
\hline 95.66 \plm 0.97 & 96.36 \plm 0.88\\
97.34 \plm 0.84 & 98.28 \plm 0.42\\
99.49 \plm 0.22 & 99.65 \plm 0.32\\
99.26 \plm 0.51 & 99.65 \plm 0.26\\
99.69 \plm 0.22 & 99.77 \plm 0.16\\
\hline 
\end{tabular}

\begin{tabular}{|c|c|}
\multicolumn{2}{c}{tropic20} \\
\hline
\hline
OvO & OvA \\
\hline 93.03 \plm 0.43 & 95.00 \plm 0.79\\
95.77 \plm 0.87 & 98.12 \plm 0.63\\
98.39 \plm 0.38 & 99.34 \plm 0.29\\
99.13 \plm 0.17 & 99.68 \plm 0.21\\
99.58 \plm 0.30 & 99.73 \plm 0.18\\
\hline 
\end{tabular}
}%
\vspace{2mm}
\resizebox{\textwidth}{!}{
\begin{tabular}{|c|c|c|}
\multicolumn{3}{c}{swedishLeaves3folds3} \\
\hline
\hline
\scriptsize{Train Prozent} & OvO & OvA \\
\hline 10 & 79.11 \plm 4.29 & 86.22 \plm 3.42\\
20 & 94.89 \plm 1.54 & 93.11 \plm 4.44\\
50 & 95.56 \plm 2.14 & 96.22 \plm 2.69\\
80 & 98.89 \plm 1.02 & 99.56 \plm 0.77\\
100 & 99.33 \plm 0.67 & 99.11 \plm 0.77\\
\hline 
\end{tabular}

\begin{tabular}{|c|c|}
\multicolumn{2}{c}{swedishLeaves3folds5} \\
\hline
\hline
OvO & OvA \\
\hline 91.07 \plm 4.03 & 88.40 \plm 5.89\\
94.27 \plm 1.51 & 94.93 \plm 1.67\\
99.60 \plm 0.40 & 99.33 \plm 0.61\\
99.47 \plm 0.92 & 99.73 \plm 0.23\\
100.00 \plm 0.00 & 99.60 \plm 0.40\\
\hline 
\end{tabular}

\begin{tabular}{|c|c|}
\multicolumn{2}{c}{swedishLeaves3folds10} \\
\hline
\hline
OvO & OvA \\
\hline 92.53 \plm 2.60 & 91.47 \plm 1.80\\
97.87 \plm 1.36 & 97.40 \plm 1.40\\
98.87 \plm 0.50 & 98.07 \plm 0.95\\
99.93 \plm 0.12 & 99.73 \plm 0.31\\
100.00 \plm 0.00 & 99.93 \plm 0.12\\
\hline 
\end{tabular}

\begin{tabular}{|c|c|}
\multicolumn{2}{c}{swedishLeaves3folds15} \\
\hline
\hline
OvO & OvA \\
\hline 95.73 \plm 1.39 & 95.82 \plm 1.56\\
99.51 \plm 0.28 & 99.02 \plm 0.43\\
99.02 \plm 0.81 & 99.20 \plm 0.48\\
99.91 \plm 0.08 & 99.87 \plm 0.13\\
99.91 \plm 0.15 & 100.00 \plm 0.00\\
\hline 
\end{tabular}
}%
\vspace{2mm}
\resizebox{\textwidth}{!}{
\begin{tabular}{|c|c|c|}
\multicolumn{3}{c}{pawara-tropic10} \\
\hline
\hline
\scriptsize{Train Prozent} & OvO & OvA \\
\hline 10 & 94.81 \plm 0.62 & 95.17 \plm 1.41\\
20 & 96.37 \plm 1.59 & 98.07 \plm 0.54\\
50 & 99.11 \plm 0.25 & 99.58 \plm 0.28\\
80 & 99.37 \plm 0.11 & 99.66 \plm 0.20\\
100 & 99.50 \plm 0.20 & 99.92 \plm 0.07\\
\hline 
\end{tabular}

\begin{tabular}{|c|c|}
\multicolumn{2}{c}{pawara-monkey10} \\
\hline
\hline
OvO & OvA \\
\hline 96.71 \plm 1.70 & 97.00 \plm 0.88\\
96.93 \plm 1.61 & 96.94 \plm 1.58\\
97.81 \plm 0.73 & 97.95 \plm 0.50\\
98.69 \plm 0.80 & 98.98 \plm 0.70\\
98.76 \plm 0.41 & 98.90 \plm 0.73\\
\hline 
\end{tabular}

\begin{tabular}{|c|c|}
\multicolumn{2}{c}{pawara-umonkey10} \\
\hline
\hline
OvO & OvA \\
\hline 90.17 \plm 3.79 & 85.42 \plm 2.77\\
95.19 \plm 1.40 & 92.12 \plm 1.28\\
95.76 \plm 1.13 & 93.94 \plm 1.68\\
96.35 \plm 1.50 & 96.64 \plm 0.92\\
95.98 \plm 1.50 & 96.28 \plm 1.20\\
\hline 
\end{tabular}
}%

\end{figure}



\subsection{InceptionV3-Pawara Scratch}
\begin{figure}[H]
\resizebox{\textwidth}{!}{
\begin{tabular}{|c|c|c|}
\multicolumn{3}{c}{agrilplant3} \\
\hline
\hline
\scriptsize{Train Prozent} & OvO & OvA \\
\hline 10 & 81.67 \plm 2.89 & 84.67 \plm 4.39\\
20 & 88.11 \plm 2.96 & 86.78 \plm 2.68\\
50 & 97.11 \plm 0.82 & 97.00 \plm 1.45\\
80 & 98.56 \plm 0.50 & 97.67 \plm 1.20\\
100 & 98.44 \plm 0.72 & 98.67 \plm 1.01\\
\hline 
\end{tabular}

\begin{tabular}{|c|c|}
\multicolumn{2}{c}{agrilplant5} \\
\hline
\hline
OvO & OvA \\
\hline 82.20 \plm 3.69 & 78.67 \plm 3.76\\
91.53 \plm 2.61 & 89.67 \plm 2.57\\
97.67 \plm 0.78 & 97.93 \plm 0.55\\
97.93 \plm 0.95 & 98.13 \plm 0.69\\
98.67 \plm 0.91 & 99.07 \plm 0.28\\
\hline 
\end{tabular}

\begin{tabular}{|c|c|}
\multicolumn{2}{c}{agrilplant10} \\
\hline
\hline
OvO & OvA \\
\hline 76.80 \plm 2.67 & 75.63 \plm 3.56\\
87.17 \plm 0.57 & 87.77 \plm 1.64\\
93.73 \plm 1.06 & 93.93 \plm 0.80\\
96.20 \plm 0.66 & 96.43 \plm 0.53\\
97.00 \plm 0.74 & 97.40 \plm 0.19\\
\hline 
\end{tabular}
}%
\vspace{2mm}
\resizebox{\textwidth}{!}{
\begin{tabular}{|c|c|c|}
\multicolumn{3}{c}{tropic3} \\
\hline
\hline
\scriptsize{Train Prozent} & OvO & OvA \\
\hline 10 & 87.38 \plm 2.30 & 89.08 \plm 4.20\\
20 & 93.45 \plm 1.63 & 93.93 \plm 2.64\\
50 & 97.45 \plm 1.08 & 98.30 \plm 0.51\\
80 & 98.91 \plm 0.66 & 98.91 \plm 0.66\\
100 & 99.39 \plm 0.43 & 98.18 \plm 0.86\\
\hline 
\end{tabular}

\begin{tabular}{|c|c|}
\multicolumn{2}{c}{tropic5} \\
\hline
\hline
OvO & OvA \\
\hline 82.64 \plm 1.33 & 81.19 \plm 1.05\\
89.62 \plm 2.47 & 88.60 \plm 4.25\\
96.37 \plm 0.76 & 96.37 \plm 1.11\\
98.69 \plm 0.41 & 98.48 \plm 0.78\\
99.42 \plm 0.49 & 98.91 \plm 0.51\\
\hline 
\end{tabular}

\begin{tabular}{|c|c|}
\multicolumn{2}{c}{tropic10} \\
\hline
\hline
OvO & OvA \\
\hline 78.79 \plm 2.28 & 73.23 \plm 3.42\\
89.67 \plm 0.94 & 86.11 \plm 2.17\\
96.67 \plm 1.25 & 96.28 \plm 0.66\\
97.96 \plm 0.43 & 98.08 \plm 0.49\\
99.06 \plm 0.32 & 98.90 \plm 0.18\\
\hline 
\end{tabular}

\begin{tabular}{|c|c|}
\multicolumn{2}{c}{tropic20} \\
\hline
\hline
OvO & OvA \\
\hline 73.54 \plm 2.26 & 70.68 \plm 0.92\\
86.13 \plm 2.48 & 85.18 \plm 1.58\\
93.95 \plm 2.91 & 95.32 \plm 0.49\\
97.40 \plm 1.34 & 97.93 \plm 0.29\\
98.24 \plm 0.95 & 98.56 \plm 0.31\\
\hline 
\end{tabular}
}%
\vspace{2mm}
\resizebox{\textwidth}{!}{
\begin{tabular}{|c|c|c|}
\multicolumn{3}{c}{swedishLeaves3folds3} \\
\hline
\hline
\scriptsize{Train Prozent} & OvO & OvA \\
\hline 10 & 33.78 \plm 0.77 & 33.33 \plm 0.00\\
20 & 33.33 \plm 0.00 & 34.67 \plm 2.31\\
50 & 38.67 \plm 9.24 & 39.78 \plm 10.59\\
80 & 85.56 \plm 14.63 & 85.56 \plm 7.19\\
100 & 96.44 \plm 3.42 & 93.78 \plm 3.29\\
\hline 
\end{tabular}

\begin{tabular}{|c|c|}
\multicolumn{2}{c}{swedishLeaves3folds5} \\
\hline
\hline
OvO & OvA \\
\hline 20.00 \plm 0.00 & 21.47 \plm 2.54\\
20.00 \plm 0.00 & 20.00 \plm 0.00\\
90.00 \plm 4.54 & 92.27 \plm 2.20\\
98.27 \plm 1.15 & 95.20 \plm 1.74\\
98.13 \plm 1.89 & 97.73 \plm 1.62\\
\hline 
\end{tabular}

\begin{tabular}{|c|c|}
\multicolumn{2}{c}{swedishLeaves3folds10} \\
\hline
\hline
OvO & OvA \\
\hline 10.13 \plm 7.55 & 10.00 \plm 0.00\\
87.60 \plm 1.60 & 82.73 \plm 3.83\\
94.07 \plm 1.10 & 93.47 \plm 2.01\\
96.00 \plm 0.92 & 95.67 \plm 1.14\\
97.67 \plm 1.30 & 97.87 \plm 0.95\\
\hline 
\end{tabular}

\begin{tabular}{|c|c|}
\multicolumn{2}{c}{swedishLeaves3folds15} \\
\hline
\hline
OvO & OvA \\
\hline 12.27 \plm 3.40 & 7.16 \plm 0.63\\
93.51 \plm 0.96 & 92.62 \plm 1.04\\
94.62 \plm 1.95 & 94.44 \plm 1.61\\
98.49 \plm 0.50 & 97.20 \plm 0.69\\
98.58 \plm 0.86 & 98.00 \plm 0.61\\
\hline 
\end{tabular}
}%
\vspace{2mm}
\resizebox{\textwidth}{!}{
\begin{tabular}{|c|c|c|}
\multicolumn{3}{c}{pawara-tropic10} \\
\hline
\hline
\scriptsize{Train Prozent} & OvO & OvA \\
\hline 10 & 74.67 \plm 3.00 & 69.16 \plm 4.27\\
20 & 86.33 \plm 1.98 & 82.68 \plm 2.16\\
50 & 96.14 \plm 0.88 & 95.12 \plm 0.59\\
80 & 97.86 \plm 0.63 & 97.73 \plm 0.38\\
100 & 98.38 \plm 1.08 & 98.38 \plm 0.54\\
\hline 
\end{tabular}

\begin{tabular}{|c|c|}
\multicolumn{2}{c}{pawara-monkey10} \\
\hline
\hline
OvO & OvA \\
\hline 51.70 \plm 4.19 & 47.23 \plm 1.22\\
66.87 \plm 2.72 & 61.83 \plm 1.17\\
81.16 \plm 2.62 & 80.37 \plm 1.78\\
90.29 \plm 2.35 & 89.85 \plm 2.47\\
93.28 \plm 1.80 & 92.26 \plm 2.26\\
\hline 
\end{tabular}

\begin{tabular}{|c|c|}
\multicolumn{2}{c}{pawara-umonkey10} \\
\hline
\hline
OvO & OvA \\
\hline 37.31 \plm 3.49 & 34.53 \plm 2.85\\
48.71 \plm 4.45 & 46.14 \plm 3.45\\
61.46 \plm 4.63 & 65.12 \plm 5.26\\
71.68 \plm 3.00 & 71.18 \plm 3.17\\
76.95 \plm 3.41 & 75.48 \plm 2.04\\
\hline 
\end{tabular}
}%

\end{figure}

\subsection{InceptionV3-Pawara Finetune}
\begin{figure}[H]
\resizebox{\textwidth}{!}{
\begin{tabular}{|c|c|c|}
\multicolumn{3}{c}{agrilplant3} \\
\hline
\hline
\scriptsize{Train Prozent} & OvO & OvA \\
\hline 10 & 99.22 \plm 0.30 & 99.78 \plm 0.30\\
20 & 99.56 \plm 0.46 & 99.56 \plm 0.46\\
50 & 99.78 \plm 0.30 & 100.00 \plm 0.00\\
80 & 99.89 \plm 0.25 & 100.00 \plm 0.00\\
100 & 100.00 \plm 0.00 & 100.00 \plm 0.00\\
\hline 
\end{tabular}

\begin{tabular}{|c|c|}
\multicolumn{2}{c}{agrilplant5} \\
\hline
\hline
OvO & OvA \\
\hline 96.13 \plm 0.77 & 95.40 \plm 1.21\\
97.93 \plm 0.98 & 97.93 \plm 1.09\\
98.73 \plm 0.28 & 99.47 \plm 0.61\\
99.40 \plm 0.49 & 99.80 \plm 0.30\\
99.47 \plm 0.61 & 99.80 \plm 0.30\\
\hline 
\end{tabular}

\begin{tabular}{|c|c|}
\multicolumn{2}{c}{agrilplant10} \\
\hline
\hline
OvO & OvA \\
\hline 91.80 \plm 0.18 & 94.27 \plm 0.42\\
94.07 \plm 1.24 & 96.57 \plm 0.55\\
97.17 \plm 0.77 & 98.50 \plm 0.49\\
97.80 \plm 0.62 & 98.47 \plm 0.71\\
98.07 \plm 0.35 & 98.90 \plm 0.32\\
\hline 
\end{tabular}
}%
\vspace{2mm}
\resizebox{\textwidth}{!}{
\begin{tabular}{|c|c|c|}
\multicolumn{3}{c}{tropic3} \\
\hline
\hline
\scriptsize{Train Prozent} & OvO & OvA \\
\hline 10 & 98.42 \plm 0.92 & 96.36 \plm 1.92\\
20 & 99.52 \plm 0.51 & 99.64 \plm 0.33\\
50 & 100.00 \plm 0.00 & 99.76 \plm 0.33\\
80 & 99.76 \plm 0.33 & 99.88 \plm 0.27\\
100 & 100.00 \plm 0.00 & 99.88 \plm 0.27\\
\hline 
\end{tabular}

\begin{tabular}{|c|c|}
\multicolumn{2}{c}{tropic5} \\
\hline
\hline
OvO & OvA \\
\hline 97.53 \plm 0.87 & 96.51 \plm 0.99\\
98.77 \plm 0.55 & 98.84 \plm 1.42\\
99.56 \plm 0.40 & 99.71 \plm 0.40\\
99.56 \plm 0.40 & 99.71 \plm 0.30\\
99.71 \plm 0.30 & 99.93 \plm 0.16\\
\hline 
\end{tabular}

\begin{tabular}{|c|c|}
\multicolumn{2}{c}{tropic10} \\
\hline
\hline
OvO & OvA \\
\hline 94.52 \plm 2.04 & 96.79 \plm 1.15\\
96.95 \plm 0.92 & 98.40 \plm 0.59\\
98.98 \plm 0.42 & 99.73 \plm 0.30\\
99.30 \plm 0.30 & 99.84 \plm 0.26\\
99.53 \plm 0.45 & 99.88 \plm 0.18\\
\hline 
\end{tabular}

\begin{tabular}{|c|c|}
\multicolumn{2}{c}{tropic20} \\
\hline
\hline
OvO & OvA \\
\hline 93.04 \plm 2.85 & 95.15 \plm 0.82\\
95.34 \plm 0.80 & 97.82 \plm 0.33\\
98.65 \plm 0.42 & 99.37 \plm 0.20\\
99.32 \plm 0.22 & 99.58 \plm 0.25\\
99.51 \plm 0.25 & 99.81 \plm 0.19\\
\hline 
\end{tabular}
}%
\vspace{2mm}
\resizebox{\textwidth}{!}{
\begin{tabular}{|c|c|c|}
\multicolumn{3}{c}{swedishLeaves3folds3} \\
\hline
\hline
\scriptsize{Train Prozent} & OvO & OvA \\
\hline 10 & 79.56 \plm 3.91 & 84.00 \plm 6.00\\
20 & 88.67 \plm 5.46 & 90.44 \plm 4.02\\
50 & 97.33 \plm 2.40 & 95.56 \plm 3.91\\
80 & 99.33 \plm 0.67 & 99.56 \plm 0.77\\
100 & 100.00 \plm 0.00 & 99.11 \plm 0.38\\
\hline 
\end{tabular}

\begin{tabular}{|c|c|}
\multicolumn{2}{c}{swedishLeaves3folds5} \\
\hline
\hline
OvO & OvA \\
\hline 86.93 \plm 7.80 & 84.00 \plm 3.49\\
94.53 \plm 0.92 & 93.07 \plm 3.80\\
99.47 \plm 0.23 & 99.60 \plm 0.40\\
99.87 \plm 0.23 & 100.00 \plm 0.00\\
99.60 \plm 0.00 & 99.73 \plm 0.23\\
\hline 
\end{tabular}

\begin{tabular}{|c|c|}
\multicolumn{2}{c}{swedishLeaves3folds10} \\
\hline
\hline
OvO & OvA \\
\hline 86.87 \plm 4.58 & 91.87 \plm 3.23\\
97.27 \plm 1.01 & 98.20 \plm 0.87\\
98.93 \plm 0.42 & 98.67 \plm 0.99\\
99.93 \plm 0.12 & 99.80 \plm 0.20\\
99.67 \plm 0.23 & 99.80 \plm 0.20\\
\hline 
\end{tabular}

\begin{tabular}{|c|c|}
\multicolumn{2}{c}{swedishLeaves3folds15} \\
\hline
\hline
OvO & OvA \\
\hline 93.96 \plm 1.95 & 95.96 \plm 1.47\\
97.51 \plm 0.31 & 99.11 \plm 0.15\\
99.24 \plm 0.20 & 99.20 \plm 0.74\\
99.56 \plm 0.47 & 99.73 \plm 0.23\\
99.73 \plm 0.23 & 99.87 \plm 0.13\\
\hline 
\end{tabular}
}%
\vspace{2mm}
\resizebox{\textwidth}{!}{
\begin{tabular}{|c|c|c|}
\multicolumn{3}{c}{pawara-tropic10} \\
\hline
\hline
\scriptsize{Train Prozent} & OvO & OvA \\
\hline 10 & 93.29 \plm 1.23 & 95.02 \plm 0.73\\
20 & 95.96 \plm 0.86 & 97.70 \plm 0.94\\
50 & 98.83 \plm 0.24 & 99.56 \plm 0.30\\
80 & 99.16 \plm 0.25 & 99.82 \plm 0.12\\
100 & 99.27 \plm 0.24 & 99.82 \plm 0.07\\
\hline 
\end{tabular}

\begin{tabular}{|c|c|}
\multicolumn{2}{c}{pawara-monkey10} \\
\hline
\hline
OvO & OvA \\
\hline 94.74 \plm 0.98 & 96.57 \plm 0.85\\
96.93 \plm 1.20 & 96.57 \plm 1.38\\
97.66 \plm 1.02 & 97.37 \plm 0.61\\
98.17 \plm 0.58 & 98.61 \plm 0.95\\
98.61 \plm 0.48 & 98.84 \plm 0.69\\
\hline 
\end{tabular}

\begin{tabular}{|c|c|}
\multicolumn{2}{c}{pawara-umonkey10} \\
\hline
\hline
OvO & OvA \\
\hline 84.98 \plm 3.81 & 86.59 \plm 4.26\\
92.26 \plm 1.85 & 92.27 \plm 2.58\\
95.18 \plm 1.46 & 94.67 \plm 0.58\\
95.40 \plm 2.07 & 96.93 \plm 0.62\\
95.84 \plm 1.66 & 97.01 \plm 1.33\\
\hline 
\end{tabular}
}%

\end{figure}





\section{PyTorch}

\subsection{ResNet-50 Scratch}
\begin{figure}[H]
\resizebox{\textwidth}{!}{
\begin{tabular}{|c|c|c|}
\multicolumn{3}{c}{agrilplant3} \\
\hline
\hline
\scriptsize{Train Prozent} & OvO & OvA \\
\hline 10 & 82.89 \plm 2.82 & 78.44 \plm 2.82\\
20 & 88.22 \plm 2.27 & 87.44 \plm 1.22\\
50 & 96.89 \plm 1.39 & 94.33 \plm 1.07\\
80 & 98.22 \plm 0.46 & 97.22 \plm 1.24\\
100 & 98.44 \plm 0.72 & 97.56 \plm 0.93\\
\hline 
\end{tabular}

\begin{tabular}{|c|c|}
\multicolumn{2}{c}{agrilplant5} \\
\hline
\hline
OvO & OvA \\
\hline 83.73 \plm 1.57 & 77.73 \plm 1.95\\
89.00 \plm 2.46 & 88.47 \plm 3.20\\
96.53 \plm 0.84 & 96.20 \plm 0.77\\
97.20 \plm 0.80 & 97.20 \plm 0.69\\
98.20 \plm 0.69 & 98.47 \plm 0.51\\
\hline 
\end{tabular}

\begin{tabular}{|c|c|}
\multicolumn{2}{c}{agrilplant10} \\
\hline
\hline
OvO & OvA \\
\hline 76.53 \plm 3.00 & 74.70 \plm 3.01\\
84.73 \plm 1.33 & 85.50 \plm 2.29\\
91.57 \plm 1.24 & 92.13 \plm 1.20\\
94.57 \plm 1.09 & 93.87 \plm 0.69\\
95.37 \plm 0.97 & 95.50 \plm 0.20\\
\hline 
\end{tabular}
}%
\vspace{2mm}
\resizebox{\textwidth}{!}{
\begin{tabular}{|c|c|c|}
\multicolumn{3}{c}{tropic3} \\
\hline
\hline
\scriptsize{Train Prozent} & OvO & OvA \\
\hline 10 & 87.02 \plm 1.92 & 88.35 \plm 1.02\\
20 & 92.23 \plm 3.30 & 92.72 \plm 1.92\\
50 & 97.57 \plm 0.60 & 96.84 \plm 1.58\\
80 & 98.42 \plm 0.92 & 98.30 \plm 0.79\\
100 & 98.54 \plm 0.92 & 98.79 \plm 0.61\\
\hline 
\end{tabular}

\begin{tabular}{|c|c|}
\multicolumn{2}{c}{tropic5} \\
\hline
\hline
OvO & OvA \\
\hline 81.12 \plm 1.86 & 78.65 \plm 3.20\\
89.04 \plm 1.74 & 87.66 \plm 2.71\\
96.22 \plm 0.75 & 97.02 \plm 1.19\\
97.68 \plm 1.32 & 98.33 \plm 0.98\\
97.97 \plm 1.16 & 98.26 \plm 1.10\\
\hline 
\end{tabular}

\begin{tabular}{|c|c|}
\multicolumn{2}{c}{tropic10} \\
\hline
\hline
OvO & OvA \\
\hline 75.73 \plm 2.78 & 72.17 \plm 3.31\\
82.78 \plm 1.68 & 82.78 \plm 1.62\\
94.01 \plm 2.42 & 93.93 \plm 0.52\\
95.54 \plm 1.74 & 96.48 \plm 0.95\\
96.83 \plm 0.67 & 97.65 \plm 0.84\\
\hline 
\end{tabular}

\begin{tabular}{|c|c|}
\multicolumn{2}{c}{tropic20} \\
\hline
\hline
OvO & OvA \\
\hline 70.87 \plm 1.69 & 69.81 \plm 1.84\\
82.16 \plm 1.62 & 83.23 \plm 0.83\\
92.74 \plm 1.11 & 93.35 \plm 1.25\\
93.58 \plm 1.99 & 95.75 \plm 0.60\\
94.52 \plm 1.54 & 96.72 \plm 0.50\\
\hline 
\end{tabular}
}%
\vspace{2mm}
\resizebox{\textwidth}{!}{
\begin{tabular}{|c|c|c|}
\multicolumn{3}{c}{swedishLeaves3folds3} \\
\hline
\hline
\scriptsize{Train Prozent} & OvO & OvA \\
\hline 10 & 78.67 \plm 1.15 & 81.11 \plm 0.77\\
20 & 85.33 \plm 2.00 & 82.67 \plm 4.67\\
50 & 89.11 \plm 4.07 & 88.44 \plm 2.78\\
80 & 96.89 \plm 0.38 & 94.44 \plm 1.02\\
100 & 96.00 \plm 1.33 & 93.56 \plm 2.69\\
\hline 
\end{tabular}

\begin{tabular}{|c|c|}
\multicolumn{2}{c}{swedishLeaves3folds5} \\
\hline
\hline
OvO & OvA \\
\hline 84.80 \plm 2.08 & 81.20 \plm 3.27\\
88.93 \plm 1.22 & 86.27 \plm 3.40\\
97.20 \plm 0.80 & 96.40 \plm 0.69\\
97.87 \plm 1.40 & 96.93 \plm 0.83\\
98.67 \plm 0.83 & 98.93 \plm 0.23\\
\hline 
\end{tabular}

\begin{tabular}{|c|c|}
\multicolumn{2}{c}{swedishLeaves3folds10} \\
\hline
\hline
OvO & OvA \\
\hline 79.27 \plm 2.47 & 79.53 \plm 6.44\\
91.73 \plm 3.06 & 93.13 \plm 2.16\\
93.27 \plm 0.31 & 92.93 \plm 2.57\\
95.47 \plm 1.21 & 93.27 \plm 0.12\\
97.87 \plm 0.76 & 98.00 \plm 1.31\\
\hline 
\end{tabular}

\begin{tabular}{|c|c|}
\multicolumn{2}{c}{swedishLeaves3folds15} \\
\hline
\hline
OvO & OvA \\
\hline 76.98 \plm 4.14 & 81.33 \plm 4.60\\
92.13 \plm 1.09 & 92.18 \plm 0.34\\
93.64 \plm 1.12 & 93.51 \plm 1.74\\
97.87 \plm 1.09 & 97.11 \plm 0.08\\
98.09 \plm 0.95 & 97.47 \plm 0.67\\
\hline 
\end{tabular}
}%
\vspace{2mm}
\resizebox{\textwidth}{!}{
\begin{tabular}{|c|c|c|}
\multicolumn{3}{c}{pawara-tropic10} \\
\hline
\hline
\scriptsize{Train Prozent} & OvO & OvA \\
\hline 10 & 69.66 \plm 2.79 & 64.60 \plm 3.03\\
20 & 82.99 \plm 1.48 & 80.20 \plm 0.27\\
50 & 92.43 \plm 1.59 & 92.61 \plm 1.27\\
80 & 96.01 \plm 0.97 & 95.10 \plm 1.29\\
100 & 96.58 \plm 1.29 & 96.82 \plm 1.25\\
\hline 
\end{tabular}

\begin{tabular}{|c|c|}
\multicolumn{2}{c}{pawara-monkey10} \\
\hline
\hline
OvO & OvA \\
\hline 48.04 \plm 4.20 & 45.41 \plm 4.71\\
58.68 \plm 3.44 & 57.59 \plm 2.63\\
73.07 \plm 5.50 & 74.81 \plm 5.48\\
85.54 \plm 2.19 & 86.20 \plm 3.33\\
88.61 \plm 2.82 & 89.79 \plm 1.25\\
\hline 
\end{tabular}

\begin{tabular}{|c|c|}
\multicolumn{2}{c}{pawara-umonkey10} \\
\hline
\hline
OvO & OvA \\
\hline 33.44 \plm 3.23 & 33.07 \plm 1.58\\
44.38 \plm 4.30 & 40.95 \plm 2.64\\
54.67 \plm 3.54 & 59.71 \plm 3.01\\
64.02 \plm 2.25 & 66.42 \plm 3.28\\
67.51 \plm 6.14 & 70.14 \plm 2.34\\
\hline 
\end{tabular}

\begin{tabular}{|c|c|}
\multicolumn{2}{c}{cifar10} \\
\hline
\hline
OvO & OvA \\
\hline 74.56 \plm 0.61 & 76.51 \plm 0.82\\
80.74 \plm 1.32 & 83.78 \plm 0.63\\
88.71 \plm 0.62 & 89.91 \plm 0.41\\
90.25 \plm 1.38 & 92.22 \plm 0.26\\
90.63 \plm 2.07 & 92.67 \plm 0.87\\
\hline 
\end{tabular}
}%

\end{figure}

\subsection{ResNet-50 Finetune}
\begin{figure}[H]
\resizebox{\textwidth}{!}{
\begin{tabular}{|c|c|c|}
\multicolumn{3}{c}{agrilplant3} \\
\hline
\hline
\scriptsize{Train Prozent} & OvO & OvA \\
\hline 10 & 99.67 \plm 0.50 & 99.44 \plm 0.39\\
20 & 99.78 \plm 0.50 & 99.56 \plm 0.25\\
50 & 99.78 \plm 0.30 & 99.89 \plm 0.25\\
80 & 100.00 \plm 0.00 & 100.00 \plm 0.00\\
100 & 100.00 \plm 0.00 & 99.78 \plm 0.30\\
\hline 
\end{tabular}

\begin{tabular}{|c|c|}
\multicolumn{2}{c}{agrilplant5} \\
\hline
\hline
OvO & OvA \\
\hline 98.27 \plm 1.19 & 98.07 \plm 0.72\\
99.27 \plm 0.64 & 99.27 \plm 0.72\\
99.67 \plm 0.24 & 99.60 \plm 0.28\\
99.73 \plm 0.28 & 99.80 \plm 0.30\\
99.80 \plm 0.18 & 99.93 \plm 0.15\\
\hline 
\end{tabular}

\begin{tabular}{|c|c|}
\multicolumn{2}{c}{agrilplant10} \\
\hline
\hline
OvO & OvA \\
\hline 94.73 \plm 0.88 & 96.20 \plm 0.57\\
96.73 \plm 0.48 & 97.30 \plm 0.77\\
97.27 \plm 0.60 & 98.53 \plm 0.36\\
98.33 \plm 0.37 & 98.70 \plm 0.61\\
98.13 \plm 0.27 & 98.80 \plm 0.48\\
\hline 
\end{tabular}
}%
\vspace{2mm}
\resizebox{\textwidth}{!}{
\begin{tabular}{|c|c|c|}
\multicolumn{3}{c}{tropic3} \\
\hline
\hline
\scriptsize{Train Prozent} & OvO & OvA \\
\hline 10 & 97.94 \plm 2.08 & 97.94 \plm 1.10\\
20 & 99.51 \plm 0.27 & 99.64 \plm 0.33\\
50 & 99.64 \plm 0.33 & 99.88 \plm 0.27\\
80 & 99.88 \plm 0.27 & 99.76 \plm 0.33\\
100 & 99.88 \plm 0.27 & 99.52 \plm 0.51\\
\hline 
\end{tabular}

\begin{tabular}{|c|c|}
\multicolumn{2}{c}{tropic5} \\
\hline
\hline
OvO & OvA \\
\hline 97.89 \plm 0.79 & 97.60 \plm 0.70\\
99.20 \plm 0.70 & 98.98 \plm 0.47\\
99.78 \plm 0.32 & 99.64 \plm 0.26\\
99.85 \plm 0.20 & 99.71 \plm 0.30\\
99.71 \plm 0.30 & 99.93 \plm 0.16\\
\hline 
\end{tabular}

\begin{tabular}{|c|c|}
\multicolumn{2}{c}{tropic10} \\
\hline
\hline
OvO & OvA \\
\hline 95.42 \plm 1.12 & 96.63 \plm 0.81\\
97.69 \plm 0.65 & 98.32 \plm 0.81\\
99.61 \plm 0.46 & 99.53 \plm 0.53\\
99.37 \plm 0.56 & 99.69 \plm 0.18\\
99.61 \plm 0.24 & 99.88 \plm 0.18\\
\hline 
\end{tabular}

\begin{tabular}{|c|c|}
\multicolumn{2}{c}{tropic20} \\
\hline
\hline
OvO & OvA \\
\hline 93.93 \plm 0.45 & 95.98 \plm 0.38\\
96.72 \plm 0.59 & 98.14 \plm 0.30\\
98.58 \plm 0.32 & 99.26 \plm 0.42\\
99.07 \plm 0.24 & 99.47 \plm 0.21\\
99.39 \plm 0.17 & 99.73 \plm 0.16\\
\hline 
\end{tabular}
}%
\vspace{2mm}
\resizebox{\textwidth}{!}{
\begin{tabular}{|c|c|c|}
\multicolumn{3}{c}{swedishLeaves3folds3} \\
\hline
\hline
\scriptsize{Train Prozent} & OvO & OvA \\
\hline 10 & 91.78 \plm 5.39 & 92.44 \plm 6.19\\
20 & 97.33 \plm 1.15 & 94.00 \plm 2.31\\
50 & 94.89 \plm 4.44 & 95.33 \plm 4.37\\
80 & 99.56 \plm 0.38 & 99.33 \plm 1.15\\
100 & 99.33 \plm 0.67 & 99.56 \plm 0.77\\
\hline 
\end{tabular}

\begin{tabular}{|c|c|}
\multicolumn{2}{c}{swedishLeaves3folds5} \\
\hline
\hline
OvO & OvA \\
\hline 94.00 \plm 2.62 & 94.93 \plm 1.29\\
95.73 \plm 3.63 & 97.20 \plm 2.50\\
99.73 \plm 0.23 & 99.73 \plm 0.46\\
99.73 \plm 0.23 & 99.73 \plm 0.46\\
99.73 \plm 0.23 & 100.00 \plm 0.00\\
\hline 
\end{tabular}

\begin{tabular}{|c|c|}
\multicolumn{2}{c}{swedishLeaves3folds10} \\
\hline
\hline
OvO & OvA \\
\hline 95.87 \plm 3.00 & 95.67 \plm 2.42\\
98.27 \plm 1.62 & 97.67 \plm 1.86\\
98.27 \plm 1.36 & 98.73 \plm 0.83\\
99.53 \plm 0.31 & 99.80 \plm 0.35\\
99.80 \plm 0.20 & 100.00 \plm 0.00\\
\hline 
\end{tabular}

\begin{tabular}{|c|c|}
\multicolumn{2}{c}{swedishLeaves3folds15} \\
\hline
\hline
OvO & OvA \\
\hline 97.38 \plm 0.20 & 97.78 \plm 0.20\\
98.93 \plm 0.48 & 98.98 \plm 0.28\\
99.29 \plm 0.28 & 99.20 \plm 0.46\\
99.73 \plm 0.27 & 99.82 \plm 0.08\\
99.73 \plm 0.13 & 99.91 \plm 0.15\\
\hline 
\end{tabular}
}%
\vspace{2mm}
\resizebox{\textwidth}{!}{
\begin{tabular}{|c|c|c|}
\multicolumn{3}{c}{pawara-tropic10} \\
\hline
\hline
\scriptsize{Train Prozent} & OvO & OvA \\
\hline 10 & 95.85 \plm 0.86 & 96.71 \plm 0.63\\
20 & 97.60 \plm 0.96 & 97.83 \plm 0.68\\
50 & 99.30 \plm 0.42 & 99.48 \plm 0.21\\
80 & 99.48 \plm 0.29 & 99.69 \plm 0.12\\
100 & 99.50 \plm 0.25 & 99.69 \plm 0.20\\
\hline 
\end{tabular}

\begin{tabular}{|c|c|}
\multicolumn{2}{c}{pawara-monkey10} \\
\hline
\hline
OvO & OvA \\
\hline 95.91 \plm 1.07 & 96.42 \plm 1.23\\
95.76 \plm 2.06 & 95.84 \plm 1.13\\
97.66 \plm 0.98 & 97.88 \plm 0.91\\
98.18 \plm 1.47 & 98.32 \plm 0.77\\
98.39 \plm 0.55 & 98.18 \plm 0.51\\
\hline 
\end{tabular}

\begin{tabular}{|c|c|}
\multicolumn{2}{c}{pawara-umonkey10} \\
\hline
\hline
OvO & OvA \\
\hline 90.67 \plm 2.11 & 86.57 \plm 1.25\\
94.53 \plm 1.20 & 89.33 \plm 4.22\\
95.18 \plm 0.79 & 94.89 \plm 1.20\\
96.13 \plm 1.49 & 95.62 \plm 0.67\\
95.54 \plm 1.84 & 95.98 \plm 1.07\\
\hline 
\end{tabular}
}%

\end{figure}


\subsection{InceptionV3 Scratch}
\begin{figure}[H]
\resizebox{\textwidth}{!}{
\begin{tabular}{|c|c|c|}
\multicolumn{3}{c}{agrilplant3} \\
\hline
\hline
\scriptsize{Train Prozent} & OvO & OvA \\
\hline 10 & 84.33 \plm 2.47 & 83.78 \plm 3.30\\
20 & 88.89 \plm 2.48 & 89.67 \plm 2.31\\
50 & 95.89 \plm 0.84 & 96.22 \plm 0.99\\
80 & 96.89 \plm 0.63 & 97.33 \plm 0.82\\
100 & 98.33 \plm 0.68 & 97.89 \plm 0.46\\
\hline 
\end{tabular}

\begin{tabular}{|c|c|}
\multicolumn{2}{c}{agrilplant5} \\
\hline
\hline
OvO & OvA \\
\hline 84.00 \plm 2.72 & 83.40 \plm 1.69\\
91.40 \plm 3.18 & 90.27 \plm 3.90\\
96.73 \plm 1.32 & 96.93 \plm 0.64\\
97.93 \plm 0.55 & 98.53 \plm 0.80\\
98.73 \plm 0.86 & 98.67 \plm 0.53\\
\hline 
\end{tabular}

\begin{tabular}{|c|c|}
\multicolumn{2}{c}{agrilplant10} \\
\hline
\hline
OvO & OvA \\
\hline 77.33 \plm 2.67 & 77.53 \plm 1.57\\
87.27 \plm 1.35 & 86.20 \plm 1.46\\
93.47 \plm 0.63 & 93.13 \plm 0.74\\
95.57 \plm 0.65 & 96.17 \plm 0.57\\
96.20 \plm 0.62 & 96.87 \plm 0.70\\
\hline 
\end{tabular}
}%
\vspace{2mm}
\resizebox{\textwidth}{!}{
\begin{tabular}{|c|c|c|}
\multicolumn{3}{c}{tropic3} \\
\hline
\hline
\scriptsize{Train Prozent} & OvO & OvA \\
\hline 10 & 88.47 \plm 2.76 & 88.47 \plm 2.94\\
20 & 94.18 \plm 1.09 & 94.05 \plm 0.26\\
50 & 96.97 \plm 1.77 & 97.57 \plm 1.05\\
80 & 98.18 \plm 0.96 & 97.82 \plm 1.33\\
100 & 99.27 \plm 0.51 & 98.42 \plm 0.69\\
\hline 
\end{tabular}

\begin{tabular}{|c|c|}
\multicolumn{2}{c}{tropic5} \\
\hline
\hline
OvO & OvA \\
\hline 80.46 \plm 2.72 & 77.19 \plm 4.57\\
90.42 \plm 3.08 & 89.11 \plm 2.28\\
97.10 \plm 1.60 & 96.01 \plm 0.92\\
98.69 \plm 0.75 & 97.82 \plm 0.36\\
98.91 \plm 0.44 & 98.84 \plm 0.65\\
\hline 
\end{tabular}

\begin{tabular}{|c|c|}
\multicolumn{2}{c}{tropic10} \\
\hline
\hline
OvO & OvA \\
\hline 76.67 \plm 1.58 & 72.33 \plm 1.91\\
86.65 \plm 1.52 & 83.99 \plm 1.71\\
96.01 \plm 0.78 & 95.58 \plm 1.24\\
97.93 \plm 0.49 & 97.46 \plm 0.68\\
98.32 \plm 1.10 & 98.59 \plm 0.51\\
\hline 
\end{tabular}

\begin{tabular}{|c|c|}
\multicolumn{2}{c}{tropic20} \\
\hline
\hline
OvO & OvA \\
\hline 72.52 \plm 1.32 & 70.01 \plm 1.80\\
85.67 \plm 0.91 & 81.92 \plm 1.90\\
94.73 \plm 0.69 & 94.62 \plm 0.93\\
97.19 \plm 0.70 & 97.67 \plm 0.44\\
98.26 \plm 0.48 & 97.99 \plm 0.58\\
\hline 
\end{tabular}
}%
\vspace{2mm}
\resizebox{\textwidth}{!}{
\begin{tabular}{|c|c|c|}
\multicolumn{3}{c}{swedishLeaves3folds3} \\
\hline
\hline
\scriptsize{Train Prozent} & OvO & OvA \\
\hline 10 & 82.22 \plm 2.14 & 75.78 \plm 4.34\\
20 & 85.78 \plm 3.79 & 88.44 \plm 2.04\\
50 & 93.33 \plm 3.06 & 94.67 \plm 2.40\\
80 & 98.67 \plm 0.67 & 98.00 \plm 1.15\\
100 & 98.22 \plm 0.77 & 97.33 \plm 1.76\\
\hline 
\end{tabular}

\begin{tabular}{|c|c|}
\multicolumn{2}{c}{swedishLeaves3folds5} \\
\hline
\hline
OvO & OvA \\
\hline 83.87 \plm 2.01 & 77.07 \plm 8.09\\
91.60 \plm 3.60 & 90.00 \plm 3.60\\
98.80 \plm 0.80 & 97.47 \plm 1.62\\
99.20 \plm 0.40 & 99.33 \plm 0.61\\
99.60 \plm 0.40 & 99.73 \plm 0.23\\
\hline 
\end{tabular}

\begin{tabular}{|c|c|}
\multicolumn{2}{c}{swedishLeaves3folds10} \\
\hline
\hline
OvO & OvA \\
\hline 85.53 \plm 4.56 & 87.60 \plm 2.27\\
95.07 \plm 2.34 & 95.73 \plm 0.64\\
97.07 \plm 0.76 & 97.07 \plm 1.36\\
99.33 \plm 0.42 & 97.60 \plm 1.40\\
99.47 \plm 0.31 & 99.20 \plm 0.40\\
\hline 
\end{tabular}

\begin{tabular}{|c|c|}
\multicolumn{2}{c}{swedishLeaves3folds15} \\
\hline
\hline
OvO & OvA \\
\hline 89.02 \plm 1.31 & 85.47 \plm 1.15\\
97.24 \plm 0.38 & 96.49 \plm 1.01\\
98.62 \plm 0.08 & 98.27 \plm 0.35\\
99.51 \plm 0.20 & 99.38 \plm 0.08\\
99.60 \plm 0.13 & 99.51 \plm 0.20\\
\hline 
\end{tabular}
}%
\vspace{2mm}
\resizebox{\textwidth}{!}{
\begin{tabular}{|c|c|c|}
\multicolumn{3}{c}{pawara-tropic10} \\
\hline
\hline
\scriptsize{Train Prozent} & OvO & OvA \\
\hline 10 & 74.64 \plm 2.71 & 69.56 \plm 2.70\\
20 & 83.75 \plm 1.27 & 81.63 \plm 2.37\\
50 & 95.17 \plm 1.21 & 94.47 \plm 0.59\\
80 & 98.20 \plm 0.31 & 97.26 \plm 0.91\\
100 & 98.46 \plm 0.72 & 98.30 \plm 0.33\\
\hline 
\end{tabular}

\begin{tabular}{|c|c|}
\multicolumn{2}{c}{pawara-monkey10} \\
\hline
\hline
OvO & OvA \\
\hline 55.19 \plm 2.56 & 50.08 \plm 1.98\\
67.58 \plm 2.91 & 65.83 \plm 3.64\\
83.35 \plm 2.76 & 83.58 \plm 2.13\\
90.58 \plm 2.05 & 91.24 \plm 0.49\\
92.33 \plm 1.49 & 92.26 \plm 1.64\\
\hline 
\end{tabular}

\begin{tabular}{|c|c|}
\multicolumn{2}{c}{pawara-umonkey10} \\
\hline
\hline
OvO & OvA \\
\hline 43.86 \plm 1.33 & 39.64 \plm 2.61\\
51.24 \plm 4.11 & 48.38 \plm 5.42\\
65.56 \plm 3.40 & 66.28 \plm 2.89\\
74.25 \plm 4.13 & 74.09 \plm 1.86\\
75.34 \plm 2.67 & 76.36 \plm 3.08\\
\hline 
\end{tabular}
}%

\end{figure}
\subsection{InceptionV3 Finetune}
\begin{figure}[H]
\resizebox{\textwidth}{!}{
\begin{tabular}{|c|c|c|}
\multicolumn{3}{c}{agrilplant3} \\
\hline
\hline
\scriptsize{Train Prozent} & OvO & OvA \\
\hline 10 & 99.78 \plm 0.30 & 99.56 \plm 0.72\\
20 & 99.44 \plm 0.79 & 99.78 \plm 0.30\\
50 & 99.67 \plm 0.30 & 99.78 \plm 0.30\\
80 & 99.89 \plm 0.25 & 99.89 \plm 0.25\\
100 & 100.00 \plm 0.00 & 100.00 \plm 0.00\\
\hline 
\end{tabular}

\begin{tabular}{|c|c|}
\multicolumn{2}{c}{agrilplant5} \\
\hline
\hline
OvO & OvA \\
\hline 95.20 \plm 2.63 & 95.67 \plm 2.51\\
97.93 \plm 0.95 & 98.20 \plm 0.77\\
98.93 \plm 0.76 & 99.53 \plm 0.30\\
99.27 \plm 0.49 & 99.67 \plm 0.41\\
99.53 \plm 0.18 & 99.53 \plm 0.30\\
\hline 
\end{tabular}

\begin{tabular}{|c|c|}
\multicolumn{2}{c}{agrilplant10} \\
\hline
\hline
OvO & OvA \\
\hline 92.47 \plm 0.52 & 94.23 \plm 0.63\\
95.03 \plm 0.93 & 96.43 \plm 0.85\\
96.90 \plm 0.51 & 98.10 \plm 0.73\\
97.73 \plm 0.58 & 98.50 \plm 0.59\\
98.10 \plm 0.89 & 98.77 \plm 0.28\\
\hline 
\end{tabular}
}%
\vspace{2mm}
\resizebox{\textwidth}{!}{
\begin{tabular}{|c|c|c|}
\multicolumn{3}{c}{tropic3} \\
\hline
\hline
\scriptsize{Train Prozent} & OvO & OvA \\
\hline 10 & 98.79 \plm 1.05 & 97.09 \plm 1.45\\
20 & 99.64 \plm 0.54 & 98.91 \plm 0.79\\
50 & 99.88 \plm 0.27 & 99.88 \plm 0.27\\
80 & 99.64 \plm 0.54 & 100.00 \plm 0.00\\
100 & 99.63 \plm 0.54 & 100.00 \plm 0.00\\
\hline 
\end{tabular}

\begin{tabular}{|c|c|}
\multicolumn{2}{c}{tropic5} \\
\hline
\hline
OvO & OvA \\
\hline 96.59 \plm 2.36 & 96.80 \plm 1.46\\
98.55 \plm 0.26 & 99.13 \plm 0.33\\
99.13 \plm 0.61 & 99.78 \plm 0.32\\
99.35 \plm 0.47 & 99.64 \plm 0.26\\
99.56 \plm 0.30 & 99.85 \plm 0.20\\
\hline 
\end{tabular}

\begin{tabular}{|c|c|}
\multicolumn{2}{c}{tropic10} \\
\hline
\hline
OvO & OvA \\
\hline 93.31 \plm 1.05 & 97.22 \plm 1.60\\
96.48 \plm 0.57 & 98.00 \plm 0.74\\
98.75 \plm 0.45 & 99.37 \plm 0.40\\
99.10 \plm 0.22 & 99.69 \plm 0.30\\
99.53 \plm 0.11 & 99.84 \plm 0.26\\
\hline 
\end{tabular}

\begin{tabular}{|c|c|}
\multicolumn{2}{c}{tropic20} \\
\hline
\hline
OvO & OvA \\
\hline 91.55 \plm 1.03 & 94.62 \plm 0.32\\
95.30 \plm 0.88 & 97.69 \plm 0.13\\
97.95 \plm 0.34 & 99.05 \plm 0.30\\
98.90 \plm 0.38 & 99.45 \plm 0.21\\
99.20 \plm 0.38 & 99.70 \plm 0.21\\
\hline 
\end{tabular}
}%
\vspace{2mm}
\resizebox{\textwidth}{!}{
\begin{tabular}{|c|c|c|}
\multicolumn{3}{c}{swedishLeaves3folds3} \\
\hline
\hline
\scriptsize{Train Prozent} & OvO & OvA \\
\hline 10 & 94.44 \plm 4.54 & 91.78 \plm 3.67\\
20 & 97.78 \plm 2.34 & 96.22 \plm 2.52\\
50 & 94.67 \plm 4.67 & 94.89 \plm 4.44\\
80 & 99.78 \plm 0.38 & 98.67 \plm 1.76\\
100 & 99.78 \plm 0.38 & 99.78 \plm 0.38\\
\hline 
\end{tabular}

\begin{tabular}{|c|c|}
\multicolumn{2}{c}{swedishLeaves3folds5} \\
\hline
\hline
OvO & OvA \\
\hline 94.27 \plm 0.61 & 95.20 \plm 1.60\\
96.40 \plm 3.86 & 97.47 \plm 1.97\\
99.47 \plm 0.46 & 99.60 \plm 0.40\\
99.87 \plm 0.23 & 99.87 \plm 0.23\\
99.87 \plm 0.23 & 100.00 \plm 0.00\\
\hline 
\end{tabular}

\begin{tabular}{|c|c|}
\multicolumn{2}{c}{swedishLeaves3folds10} \\
\hline
\hline
OvO & OvA \\
\hline 95.53 \plm 2.39 & 95.27 \plm 1.10\\
97.47 \plm 0.70 & 97.47 \plm 1.67\\
98.60 \plm 0.72 & 98.20 \plm 1.39\\
99.93 \plm 0.12 & 99.87 \plm 0.12\\
99.93 \plm 0.12 & 100.00 \plm 0.00\\
\hline 
\end{tabular}

\begin{tabular}{|c|c|}
\multicolumn{2}{c}{swedishLeaves3folds15} \\
\hline
\hline
OvO & OvA \\
\hline 97.60 \plm 0.67 & 97.24 \plm 0.20\\
98.58 \plm 0.41 & 98.93 \plm 0.48\\
99.16 \plm 0.77 & 99.64 \plm 0.15\\
99.64 \plm 0.20 & 99.96 \plm 0.08\\
99.73 \plm 0.13 & 100.00 \plm 0.00\\
\hline 
\end{tabular}
}%
\vspace{2mm}
\resizebox{\textwidth}{!}{
\begin{tabular}{|c|c|c|}
\multicolumn{3}{c}{pawara-tropic10} \\
\hline
\hline
\scriptsize{Train Prozent} & OvO & OvA \\
\hline 10 & 92.17 \plm 0.81 & 95.49 \plm 0.39\\
20 & 95.02 \plm 1.33 & 98.09 \plm 0.69\\
50 & 98.28 \plm 0.37 & 99.48 \plm 0.26\\
80 & 99.01 \plm 0.15 & 99.77 \plm 0.11\\
100 & 99.37 \plm 0.11 & 99.76 \plm 0.11\\
\hline 
\end{tabular}

\begin{tabular}{|c|c|}
\multicolumn{2}{c}{pawara-monkey10} \\
\hline
\hline
OvO & OvA \\
\hline 95.83 \plm 1.19 & 95.98 \plm 1.53\\
95.98 \plm 1.22 & 97.22 \plm 1.09\\
97.23 \plm 1.12 & 98.32 \plm 0.55\\
98.10 \plm 0.97 & 98.91 \plm 0.58\\
98.54 \plm 0.37 & 99.05 \plm 0.75\\
\hline 
\end{tabular}

\begin{tabular}{|c|c|}
\multicolumn{2}{c}{pawara-umonkey10} \\
\hline
\hline
OvO & OvA \\
\hline 89.37 \plm 4.02 & 86.58 \plm 3.34\\
92.78 \plm 2.81 & 92.62 \plm 2.18\\
93.94 \plm 0.99 & 95.32 \plm 0.94\\
94.45 \plm 1.42 & 96.87 \plm 1.18\\
94.45 \plm 1.99 & 96.35 \plm 0.87\\
\hline 
\end{tabular}
}%

\end{figure}

\restoregeometry %Normale Seitenränder wiederherstellen
\let\@makechapterhead\savedchap % Platz vor Chapter-Überschrift auf Standard zurücksetzen

\chapter{Scatterplots}
\label{ch:Anhang_Scatterplots}

\section{Scatterplots der Accuracies}
\label{ch:Anhang_ScatterplotsAccuracies}
\begin{figure}[H]
\begin{adjustbox}{width=\textwidth, center}
\includesvg{img/3_I-S}
\end{adjustbox}
\caption{Scatterplot der erzielten Accuracies mit beiden Klassifikationsschemata in allen 3 Frameworks für Inception Scratch.}
\label{fig:ScatterplotIS}
\end{figure}
\begin{figure}[H]
\begin{adjustbox}{width=\textwidth, center}
\includesvg{img/3_I-F}
\end{adjustbox}
\caption{Scatterplot der erzielten Accuracies mit beiden Klassifikationsschemata in allen 3 Frameworks für Inception Finetune.}
\label{fig:ScatterplotIF}
\end{figure}
\begin{figure}[H]
\begin{adjustbox}{width=\textwidth, center}
\includesvg{img/3_IP-S}
\end{adjustbox}
\caption{Scatterplot der erzielten Accuracies mit beiden Klassifikationsschemata unter Verwendung von beiden TensorFlow \cite{tensorflow} Versionen für Inception-Pawara Scratch.}
\label{fig:ScatterplotIPS}
\end{figure}
\begin{figure}[H]
\begin{adjustbox}{width=\textwidth, center}
\includesvg{img/3_IP-F}
\end{adjustbox}
\caption{Scatterplot der erzielten Accuracies mit beiden Klassifikationsschemata unter Verwendung von beiden TensorFlow \cite{tensorflow} Versionen für Inception-Pawara Finetune.}
\label{fig:ScatterplotIPF}
\end{figure}


\section{Scatterplots für die Trainingsdauer}
\label{ch:Anhang_ScatterplotsDauer}

\begin{figure}[H]
\begin{adjustbox}{width=\textwidth, center}
\includesvg{img/3_R-F-dauer}
\end{adjustbox}
\caption{Scatterplot der benötigten Trainingszeit in Minuten für beide Klassifikationsschemata in allen 3 Frameworks für Resnet Finetune.}
\label{fig:ScatterplotRF-dauer}
\end{figure}
\begin{figure}[H]
\begin{adjustbox}{width=\textwidth, center}
\includesvg{img/3_I-S-dauer}
\end{adjustbox}
\caption{Scatterplot der benötigten Trainingszeit in Minuten für beide Klassifikationsschemata in allen 3 Frameworks für Inception Scratch.}
\label{fig:ScatterplotIS-dauer}
\end{figure}
\begin{figure}[H]
\begin{adjustbox}{width=\textwidth, center}
\includesvg{img/3_I-F-dauer}
\end{adjustbox}
\caption{Scatterplot der benötigten Trainingszeit in Minuten für beide Klassifikationsschemata in allen 3 Frameworks für Inception Finetune.}
\label{fig:ScatterplotIF-dauer}
\end{figure}
\begin{figure}[H]
\begin{adjustbox}{width=\textwidth, center}
\includesvg{img/3_IP-S-dauer}
\end{adjustbox}
\caption{Scatterplot der benötigten Trainingszeit in Minuten für beide Klassifikationsschemata unter Verwendung von beiden TensorFlow \cite{tensorflow} Versionen für Inception-Pawara Scratch.}
\label{fig:ScatterplotIPS-dauer}
\end{figure}
\begin{figure}[H]
\begin{adjustbox}{width=\textwidth, center}
\includesvg{img/3_IP-F-dauer}
\end{adjustbox}
\caption{Scatterplot der benötigten Trainingszeit in Minuten für beide Klassifikationsschemata unter Verwendung von beiden TensorFlow \cite{tensorflow} Versionen für Inception-Pawara Finetune.}
\label{fig:ScatterplotIPF-dauer}
\end{figure}